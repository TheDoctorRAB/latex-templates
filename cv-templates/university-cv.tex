%@TheDoctorRAB
%resume
%
%%%%% general 
\documentclass[10pt,letterpaper]{article}
\usepackage[lmargin=1in,rmargin=1in,tmargin=1in,bmargin=1in]{geometry}
\usepackage[pagewise]{lineno} %line numbering
\usepackage{setspace}
\usepackage{ulem} %strikethrough - do not \sout{\cite{}}
\usepackage[pdftex,dvipsnames]{xcolor,colortbl} %change font color
\usepackage{graphicx}
\usepackage{filecontents}
\usepackage{tablefootnote}
\usepackage{footnotehyper}
\usepackage{float}
%\usepackage{mypythonhighlight,verbatim}
%\usepackage{subfig}
\usepackage[yyyymmdd]{datetime} %date format
\renewcommand{\dateseparator}{.}
\graphicspath{{$TEXIMG/}} %path to graphics
\setcounter{secnumdepth}{5} %set subsection to nth level
\usepackage{needspace}
\usepackage[stable,hang,flushmargin]{footmisc} %footnotes in section titles and no indent; standard.bst
\usepackage[inline]{enumitem}
\usepackage{boldline}
\usepackage{makecell}
\usepackage{booktabs}
\usepackage{amssymb}
\usepackage{gensymb}
\usepackage{amsmath,nicefrac}
\usepackage{physics}
\usepackage{lscape}
\usepackage{array}
\usepackage{chngcntr}
\usepackage{hyperref}
\hypersetup{colorlinks,linkcolor=black,citecolor=black,urlcolor=blue} 
\usepackage{sectsty}
\usepackage{textcomp}
\usepackage{lastpage}
\usepackage{xargs} %for \newcommandx
\usepackage[colorinlistoftodos,prependcaption,textsize=tiny]{todonotes} %makes colored boxes for commenting
\usepackage{soul}
\usepackage{color}
\usepackage{marginnote}
\usepackage[figure,table]{totalcount}
\usepackage[capitalise]{cleveref}
\usepackage{parskip} %\noindent
\usepackage[font=itshape]{quoting}
%%%%%


%%%%% fonts
\usepackage{times}
%arial - uncomment next two lines
%\usepackage{helvet}
%\renewcommand{\familydefault}{\sfdefault}
%%%%%


%%%%% references
\usepackage{bibentry} %lists references without \citep
\usepackage[round,semicolon]{natbib} %for (Borrelli 2021; Clooney 2019) - standard.bst 
%\usepackage[numbers,sort&compress]{natbib} %for [1-3] - nsf.bst, neup.bst
%\setlength{\bibsep}{7pt} %sets space between references
%\renewcommand{\bibsection}{} %suppresses large 'references' heading
%\renewcommand\bibpreamble{\vspace{\baselineskip}} %sets spacing after heading if not using default references heading
%%%%%


%%%%% tables and figures
\usepackage{longtable} %need to put label at top under caption then \\ - use spacing
\usepackage{tablefootnote}
\usepackage{tabularx}
\usepackage{multirow}
\usepackage{tabto} %general tabbed spacing
\usepackage{pdfpages}
\usepackage{wrapfig} %wraps figures around text
\setlength{\intextsep}{0.00mm}
\setlength{\columnsep}{1.00mm}
\usepackage[singlelinecheck=false,labelfont=bf]{caption}
\usepackage{subcaption}
\captionsetup[table]{justification=justified,skip=5pt,labelformat={default},labelsep=period,name={Table}} %sets a space after table caption
\captionsetup[figure]{justification=justified,skip=5pt,labelformat={default},labelsep=period,name={Figure}} %sets space above caption, 'figure' format
\captionsetup[wrapfigure]{justification=centering,aboveskip=0pt,belowskip=0pt,labelformat={default},labelsep=period,name={Fig.}} %sets space above caption, 'figure' format
\captionsetup[wraptable]{justification=centering,aboveskip=0pt,belowskip=0pt,labelformat={default},labelsep=period,name={Table}} %sets space above caption, 'figure' format
%%%%%


%%%%% table alignments
\newcolumntype{L}[1]{>{\raggedright\let\newline\\\arraybackslash\hspace{0pt}}p{#1}} %uses \raggedright with m,p{} in table column
\newcolumntype{C}[1]{>{\centering\let\newline\\\arraybackslash\hspace{0pt}}p{#1}} %uses \raggedright with m,p{} in table column
\newcolumntype{R}[1]{>{\raggedleft\let\newline\\\arraybackslash\hspace{0pt}}p{#1}} %uses \raggedright with m,p{} in table column
%%%%%


%%%%% heading formatting
\usepackage[title,titletoc]{appendix}
\usepackage[acronym,nomain,nonumberlist]{glossaries}
\makenoidxglossaries
\usepackage{titlesec,titletoc}
%\renewcommand{\thepart}{ARTICLE \Roman{part}} %puts the label into the command so \thelabel will carry through
%\renewcommand{\thesection}{\arabic{section}} %puts the label into the command so \thelabel will carry through
%\titleformat{\part}{\normalfont\Large\bfseries\filcenter}{\thepart}{}{}[]
%\titlespacing*\part{0pt}{0.95\baselineskip}{0.75\baselineskip}

\titleformat{\section}[hang]{\normalfont\normalsize\bfseries}{\MakeUppercase{\sectiontitlename}\thesection}{}{\MakeUppercase}[]
\titlespacing*\part{0pt}{0\baselineskip}{0\baselineskip}

%\titleformat{\section}{\normalfont\normalsize\bfseries\scshape}{\thesection}{}{}[]
%\titlespacing*\section{0pt}{0.95\baselineskip}{0.15\baselineskip}

\titleformat{\subsection}{\normalfont\Large\bfseries\scshape}{\thesubsection}{}{}[]
\titlespacing*\subsection{0pt}{0.55\baselineskip}{0.10\baselineskip}

\titleformat{\subsubsection}[runin]{\normalfont\large\bfseries\scshape}{\thesubsubsection}{}{}[]
\titlespacing*\subsubsection{0pt}{0.30\baselineskip}{0.05\baselineskip}

\titleformat{\paragraph}[runin]{\normalfont\normalsize\itshape}{\theparagraph}{7pt}{}[]
\titlespacing*\paragraph{0pt}{0.25\baselineskip}{0.55\baselineskip}

\titleformat{\subparagraph}[runin]{\normalfont\normalsize}{\thesubparagraph}{-1em}{}[]
\titlespacing*\subparagraph{0pt}{0.15\baselineskip}{0.05\baselineskip}
%\titleformat{\paragraph}[hang]{\normalfont\normalsize\bfseries}{\theparagraph}{5pt}{}[]
%\titlespacing*\paragraph{0pt}{0.50\baselineskip}{0.25\baselineskip}
%\titleformat{\subparagraph}[runin]{\normalfont\normalsize\itshape}{\thesubparagraph}{-1em}{}[.]
%\titlespacing*\subparagraph{0pt}{0.40\baselineskip}{0.20\baselineskip}
%%%%%


%%%%% subject matter formatting
\newcommand{\sj}[1]{%
 {
  \bfseries\MakeUppercase{#1}:%
 }%
}
%%%%%


%%%%% editing
\newcommand{\edit}[1]{\textcolor{blue}{#1}} %shortcut for changing font color on revised text
\newcommand{\fn}[1]{\footnote{#1}} %shortcut for footnote tag
\newcommand*\sq{\mathbin{\vcenter{\hbox{\rule{.3ex}{.3ex}}}}} %makes a small square as a separator $\sq$
%\newcommand{\sk}[1]{\sout{#1}} %shortcut for default strikethrough - do not sk through citep
\newcommand\sk{\bgroup\markoverwith{\textcolor{red}{\rule[0.5ex]{1pt}{1pt}}}\ULon} %strikethrough with red line; not in \section{}
%\st{} does strikethrough using soul package but does not like acronyms
%%%%%


%%%%% colors
%http://latexcolor.com/
%https://en.wikibooks.org/wiki/LaTeX/Colors#:~:text=black%2C%20blue%2C%20brown%2C%20cyan,be%20available%20on%20all%20systems.
\definecolor{aliceblue}{rgb}{0.94, 0.97, 1.0}
\definecolor{antiquewhite}{rgb}{0.98, 0.92, 0.84}
\definecolor{lightmauve}{rgb}{0.86, 0.82, 1.0}
\definecolor{brilliantlavender}{rgb}{0.96, 0.73, 1.0}
\definecolor{brandeisblue}{rgb}{0.0, 0.44, 1.0}
\definecolor{darkmidnightblue}{rgb}{0.0, 0.2, 0.4}
%%%%%


%%%%% acronyms
\newcommand{\acf}{\acrfull} %full acronym
\newcommand{\acl}{\acrlong} %long acronym
\newcommand{\acs}{\acrshort} %short acronym

\newcommand{\acfp}{\acrfullpl} %full acronym plural
\newcommand{\aclp}{\acrlongpl} %long acronym plural
\newcommand{\acsp}{\acrshortpl} %short acronym plural
%%%%%


%%%%% header and footer
\usepackage{fancyhdr} 
\pagestyle{fancy}
\fancyhf{} %page number on bottom right
\renewcommand{\headrulewidth}{0pt} %set line thickness in header; uncomment as is to remove line
\lhead{\scriptsize BORRELLI, Robert Angelo}
%\chead{\scriptsize \textit{Vita}}
\rhead{\scriptsize Page \thepage} 
%\lfoot{\scriptsize }
%\cfoot{\scriptsize } 
%\rfoot{\thepage} %page number
%%%%%


%%%%% acronyms
\newacronym{nrs}{NRHES}{Nuclear Renewable Hybrid Energy System}
\newacronym{ahp}{AHP}{Analytical Hierarchy Process}
\newacronym{inl}{INL}{Idaho National Laboratory}
\newacronym{orl}{ORNL}{Oak Ridge National Laboratory}
\newacronym{anl}{ANL}{Argonne National Laboratory}
\newacronym{smr}{SMR}{Small Modular Reactor}
\newacronym{nus}{NuScale}{NuScale Power, LLC}
\newacronym{epri}{EPRI}{Electric Power Research Institute}
\newacronym{ci}{CI}{Consistency Index}
\newacronym{cr}{CR}{Consistency Ratio}
\newacronym{lwr}{LWR}{Light Water Reactor}
\newacronym{oer}{OER}{Online Educational Resource}
\newacronym{cps}{CPS}{Cyber-Physical Systems}
\newacronym{wsc}{WSC}{Western Services Corporation}
\newacronym{hsl}{HSSL}{Human System Simulation Laboratory}
\newacronym{roi}{ROI}{Return on Investment}
\newacronym{mwe}{MWe}{Megawatts-electric}
\newacronym{sca}{SCADA}{Supervisory Control and Data Acquisition}
\newacronym{udp}{UDP}{User Datagram Protocol}
\newacronym{plc}{PLC}{Programmable Logic Controller}
\newacronym{khp}{KHNP}{Korean Hydro \& Nuclear Power Co., Ltd}
\newacronym{jcp}{JCPOA}{Joint Comprehensive Plan of Action}
\newacronym{mim}{MITM}{Man in the Middle}
\newacronym{tcp}{TCP/IP}{Transmission Control Protocol/Internet Protocol}
\newacronym{pra}{PRA}{Probabilistic Risk Assessment}
\newacronym{cs}{CS}{Critical System}
\newacronym{loc}{LOCA}{Loss of Coolant Accident}
\newacronym{pha}{PHA}{Preliminary Hazards Analysis}
\newacronym{imu}{IMUNES}{Integrated Multiprotocol Network Emulator}
\newacronym{ccc}{CCC}{Computing Community Consortium}
\newacronym{neu}{NEUP}{Nuclear Energy University Program}
\newacronym{doe}{DOE}{United States Department of Energy}
\newacronym{nei}{NEI}{Nuclear Energy Institute}
\newacronym{nit}{NITRD}{Networking Information Technology Research \& Development Program}
\newacronym{rcs}{RCS}{Reactor Cooling System}
\newacronym{nic}{NIC}{Network Interface Card}
\newacronym{pc}{PCAP}{Packet Capture File}
\newacronym{csi}{CSIS}{Center for Strategic \& International Studies}
\newacronym{ste}{STEM}{Science, Technology, Engineering, and Mathematics}
\newacronym{pi}{PI}{Principal Investigator}
\newacronym{ans}{ANS}{American Nuclear Society}
\newacronym{sac}{SACNAS}{Society for the Advancement of Chicanos/Hispanics and Native Americans in Science}
\newacronym{ore}{ORED}{Office of Research and Economic Development}
\newacronym{rfd}{RFD}{Office of Research and Faculty Development}
\newacronym{nbe}{NSBE}{National Society of Black Engineers}
\newacronym{inp}{INPO}{Institute of Nuclear Power Operations}
\newacronym{nan}{NANTeL}{National Academy for Nuclear Training e-Learning}
\newacronym{cda}{CDA}{Critical Digital Assets}
\newacronym{nca}{NCA-CD}{National Center of Excellence in Cyber Defense}
\newacronym{ttx}{TTX}{Tabletop Exercise}
\newacronym{cie}{CIE}{Cyber-Informed Engineering}
\newacronym{aas}{AAS}{Adversary-As-A-Service}
\newacronym{iff}{UIIF}{Idaho Falls Center for Higher Education}
\newacronym{fy}{FY}{Fiscal Year}
\newacronym{eps}{EPSCoR}{Established Program to Stimulate Competitive Research}
\newacronym{nas}{NAS}{National Academies}
\newacronym{brc}{BRC}{Blue Ribbon Commission on America's Nuclear Future}
\newacronym{hlw}{HLW}{High-Level Radioactive Waste}
\newacronym{cbs}{CBS}{Consent-Based Siting}
\newacronym{wpp}{WIPP}{Waste Isolation Pilot Plant}
\newacronym{cfr}{CFR}{Code of Federal Regulations}
\newacronym{gao}{GAO}{Government Accountability Office}
\newacronym{npp}{NPP}{Nuclear Power Plant}
\newacronym{ump}{UAMPS}{Utah Associated Municipal Power Systems}
\newacronym{nrc}{NRC}{Nuclear Regulatory Commission}
\newacronym{nerc}{NERC}{North American Electric Reliability Corporation}
\newacronym{htse}{HTSE}{High Temperature Steam Electrolysis}
\newacronym{eia}{EIA}{U.S. Energy Information Administration}
\newacronym{lms}{LMS}{Learning Management System}
\newacronym{nsf}{NSF}{National Science Foundation}
\newacronym{cae}{CAES}{Center for Advanced Energy Studies}
\newacronym{pwr}{PWR}{Pressurized Water Reactor}
\newacronym{bwr}{BWR}{Boiling Water Reactor}
\newacronym{ic}{I\&C}{Instrumentation \& Controls}
\newacronym{ics}{ICS}{Industrial Control Systems}
\newacronym{ip}{IP}{Internet Protocol}
\newacronym{tva}{TVA}{Tennessee Valley Authority}
\newacronym{vfd}{VFD}{Variable Frequency Drive}
\newacronym{onl}{ORNL}{Oak Ridge National Laboratory}
\newacronym{dos}{DDoS}{Distributed Denial of Service}
\newacronym{dnp}{DNP3}{Distributed Network Protocol 3}
\newacronym{dra}{DRA}{Dynamic Risk Assessment}
\newacronym{hmi}{HMI}{Human Machine Interface}
\newacronym{bol}{BOL}{Beginning-of-Life}
\newacronym{eol}{EOL}{End-of-Life}
\newacronym{mol}{MOL}{Middle-of-Life}
\newacronym{con}{IC}{Initial Condition}
\newacronym{dc}{DC}{Direct-Current}
\newacronym{ac}{AC}{Alternating-Current}
\newacronym{snl}{SNL}{Sandia National Laboratory}
\newacronym{cds}{CRDS}{Control Rod Drive System}
\newacronym{cdm}{CRDM}{Control Rod Drive Mechanism}
\newacronym{fma}{FMEA}{Failure Modes \& Effects Analysis}
\newacronym{rpn}{RPN}{Risk Priority Number}
\newacronym{scr}{SCR}{silicon controller rectifier}
\newacronym{hvc}{HVAC}{Heating, Ventilation \& Air Conditioning}
\newacronym{ttb}{TTB}{Time-to-Boil}
\newacronym{sis}{SIS}{Safety Instrumented System}
\newacronym{cis}{CISF}{Consolidated Interim Storage Facility}
\newacronym{iea}{IEA}{International Energy Agency}
\newacronym{snf}{SNF}{Spent Nuclear Fuel}
\newacronym{cic}{CISAC}{Center for Interantional Security and Cooperation}
\newacronym{usa}{US}{United states}
\newacronym{nwa}{NWPA}{Nuclear Waste Policy Act}
\newacronym{one}{NE}{Office of Nuclear Energy}
\newacronym{cgc}{CGC}{Common Ground Consortium}
\newacronym{cop}{CoPL}{Communities of Place}
\newacronym{cor}{CoPR}{Communities of Practice}
\newacronym{coh}{CoH}{Communities of Historical Experience}
\newacronym{ej}{EJ}{Environmental Justice}
\newacronym{ite}{ITEK}{Indigenous Traditional Technical Ecological Knowledge}
\newacronym{iae}{IAEA}{International Atomic Energy Agency}
\newacronym{rfi}{RFI}{Request for Information}
\newacronym{sta}{STAND}{Siting Tool for Advanced Nuclear Development}
\newacronym{ors}{OR-SAGE}{Oak Ridge Siting Analysis for Power Generation Expansion}
\newacronym{gis}{GIS}{Geographic Information System}
\newacronym{mtu}{MTU}{Metric Ton of Uranium}
\newacronym{mth}{MTHM}{Metric Ton of Heavy Metal}
\newacronym{isa}{ISA}{Idaho Settlement Agreement}
\newacronym{tru}{TRU}{Transuranic Waste}
\newacronym{cej}{CEJEST}{Climate and Economic Justice Screening Tool}
\newacronym{eja}{EJAtlas}{Global Atlas of Environmental Justice}
\newacronym{ejs}{EJScreen}{Environmental Justice Screening and Mapping Tool}
\newacronym{fpz}{FPTZ}{Fastest Path to Zero}
\newacronym{tra}{WebTRAGIS}{Web-Based Transportation Routing Analysis Geographic Information System}
\newacronym{lit}{LITE}{Land-area Identification, Tagging, and Exploration}
\newacronym{str}{START}{Stakeholder Tool for Assessing Radioactive Transportation}
\newacronym{nri}{NRIC}{National Reactor Innovation Center}
\newacronym{sec}{SEC}{Site Evaluation Criteria}
\newacronym{ceq}{CEQ}{Council on Environmental Quality}
\newacronym{epa}{EPA}{Environmental Protection Agency}
\newacronym{unf}{UNF-ST\&DARDS}{Used Nuclear Fuel-Storage, Transportation \& Disposal Analysis Resource and Data System}
\newacronym{gui}{GUI}{Graphical User Interface}
\newacronym{fem}{FEMA}{Federal Emergency Management Agency}
\newacronym{rap}{RAPT}{Resilience Analysis and Planning Tool}
\newacronym{gid}{GEOID}{Geographic Identifier}
\newacronym{icr}{I-CREWS}{Idaho Community-Engaged Resilience for Energy-Water Systems}
\newacronym{ew}{E-W}{Energy-Water}
\newacronym{ml}{ML}{Machine Learning}
\newacronym{set}{SETS}{Socio-Environmental Technological System}
\newacronym{rii}{RII}{Research Infrastructure Improvement}
\newacronym{aoo}{AOO}{Anticipated Operational Occurrence}
\newacronym{dbe}{DBE}{Design Basis Event}
\newacronym{bbe}{BDBE}{Beyond Design Basis Event}
\newacronym{fta}{FTA}{Fault Tree Analysis}
\newacronym{eta}{ETA}{Event Tree Analysis}
\newacronym{haz}{HAZOP}{Hazard and Operability Analysis}
\newacronym{onr}{ONR}{Office of Naval Research}
\newacronym{ui}{UI}{University of Idaho}
\newacronym{isu}{ISU}{Idaho State University}
\newacronym{bsu}{BSU}{Boise State University}
\newacronym{skp}{SKP}{Senior Key Personnel}
\newacronym{bpc}{BPC}{Broadening Participation in Computing}
\newacronym{epi}{EPI}{Energy Policy Institute}
\newacronym{ym}{YM}{Yucca Mountain}
\newacronym{nwf}{NWF}{Nuclear Waste Fund}
\newacronym{aea}{AEA}{Atomic Energy Act}
\newacronym{aec}{AEC}{Atomic Energy Commission}
\newacronym{un}{UN}{United Nations}
\newacronym{des}{DES}{Discrete Event Simulation}
\newacronym{nem}{NEIM}{Nuclear Engineering \& Industrial Management}
\newacronym{erd}{ERDA}{Energy Research and Development Administration}
\newacronym{rd}{R\&D}{Research \& Development}
\newacronym{nts}{NTS}{Nevada Test Site}
\newacronym{nrm}{NRMSAC}{Nevada Radioactive Materials Storage Advisory Committee}
\newacronym{tmi}{TMI}{Three Mile Island}
\newacronym{nwt}{NWTS}{National Waste Terminal Storage}
\newacronym{irg}{IRG}{Interagency Review Group on Nuclear Waste Management}
\newacronym{kwh}{kWh}{Kilowatt-Hour}
\newacronym{eis}{EIS}{Environmental Impact Statement}
%\newacronym{}{}{}
%Add \newacronym{nrs}{NRHES}{Nuclear Renewable Hybrid Energy System}
\newacronym{ahp}{AHP}{Analytical Hierarchy Process}
\newacronym{inl}{INL}{Idaho National Laboratory}
\newacronym{orl}{ORNL}{Oak Ridge National Laboratory}
\newacronym{anl}{ANL}{Argonne National Laboratory}
\newacronym{smr}{SMR}{Small Modular Reactor}
\newacronym{nus}{NuScale}{NuScale Power, LLC}
\newacronym{epri}{EPRI}{Electric Power Research Institute}
\newacronym{ci}{CI}{Consistency Index}
\newacronym{cr}{CR}{Consistency Ratio}
\newacronym{lwr}{LWR}{Light Water Reactor}
\newacronym{oer}{OER}{Online Educational Resource}
\newacronym{cps}{CPS}{Cyber-Physical Systems}
\newacronym{wsc}{WSC}{Western Services Corporation}
\newacronym{hsl}{HSSL}{Human System Simulation Laboratory}
\newacronym{roi}{ROI}{Return on Investment}
\newacronym{mwe}{MWe}{Megawatts-electric}
\newacronym{sca}{SCADA}{Supervisory Control and Data Acquisition}
\newacronym{udp}{UDP}{User Datagram Protocol}
\newacronym{plc}{PLC}{Programmable Logic Controller}
\newacronym{khp}{KHNP}{Korean Hydro \& Nuclear Power Co., Ltd}
\newacronym{jcp}{JCPOA}{Joint Comprehensive Plan of Action}
\newacronym{mim}{MITM}{Man in the Middle}
\newacronym{tcp}{TCP/IP}{Transmission Control Protocol/Internet Protocol}
\newacronym{pra}{PRA}{Probabilistic Risk Assessment}
\newacronym{cs}{CS}{Critical System}
\newacronym{loc}{LOCA}{Loss of Coolant Accident}
\newacronym{pha}{PHA}{Preliminary Hazards Analysis}
\newacronym{imu}{IMUNES}{Integrated Multiprotocol Network Emulator}
\newacronym{ccc}{CCC}{Computing Community Consortium}
\newacronym{neu}{NEUP}{Nuclear Energy University Program}
\newacronym{doe}{DOE}{United States Department of Energy}
\newacronym{nei}{NEI}{Nuclear Energy Institute}
\newacronym{nit}{NITRD}{Networking Information Technology Research \& Development Program}
\newacronym{rcs}{RCS}{Reactor Cooling System}
\newacronym{nic}{NIC}{Network Interface Card}
\newacronym{pc}{PCAP}{Packet Capture File}
\newacronym{csi}{CSIS}{Center for Strategic \& International Studies}
\newacronym{ste}{STEM}{Science, Technology, Engineering, and Mathematics}
\newacronym{pi}{PI}{Principal Investigator}
\newacronym{ans}{ANS}{American Nuclear Society}
\newacronym{sac}{SACNAS}{Society for the Advancement of Chicanos/Hispanics and Native Americans in Science}
\newacronym{ore}{ORED}{Office of Research and Economic Development}
\newacronym{rfd}{RFD}{Office of Research and Faculty Development}
\newacronym{nbe}{NSBE}{National Society of Black Engineers}
\newacronym{inp}{INPO}{Institute of Nuclear Power Operations}
\newacronym{nan}{NANTeL}{National Academy for Nuclear Training e-Learning}
\newacronym{cda}{CDA}{Critical Digital Assets}
\newacronym{nca}{NCA-CD}{National Center of Excellence in Cyber Defense}
\newacronym{ttx}{TTX}{Tabletop Exercise}
\newacronym{cie}{CIE}{Cyber-Informed Engineering}
\newacronym{aas}{AAS}{Adversary-As-A-Service}
\newacronym{iff}{UIIF}{Idaho Falls Center for Higher Education}
\newacronym{fy}{FY}{Fiscal Year}
\newacronym{eps}{EPSCoR}{Established Program to Stimulate Competitive Research}
\newacronym{nas}{NAS}{National Academies}
\newacronym{brc}{BRC}{Blue Ribbon Commission on America's Nuclear Future}
\newacronym{hlw}{HLW}{High-Level Radioactive Waste}
\newacronym{cbs}{CBS}{Consent-Based Siting}
\newacronym{wpp}{WIPP}{Waste Isolation Pilot Plant}
\newacronym{cfr}{CFR}{Code of Federal Regulations}
\newacronym{gao}{GAO}{Government Accountability Office}
\newacronym{npp}{NPP}{Nuclear Power Plant}
\newacronym{ump}{UAMPS}{Utah Associated Municipal Power Systems}
\newacronym{nrc}{NRC}{Nuclear Regulatory Commission}
\newacronym{nerc}{NERC}{North American Electric Reliability Corporation}
\newacronym{htse}{HTSE}{High Temperature Steam Electrolysis}
\newacronym{eia}{EIA}{U.S. Energy Information Administration}
\newacronym{lms}{LMS}{Learning Management System}
\newacronym{nsf}{NSF}{National Science Foundation}
\newacronym{cae}{CAES}{Center for Advanced Energy Studies}
\newacronym{pwr}{PWR}{Pressurized Water Reactor}
\newacronym{bwr}{BWR}{Boiling Water Reactor}
\newacronym{ic}{I\&C}{Instrumentation \& Controls}
\newacronym{ics}{ICS}{Industrial Control Systems}
\newacronym{ip}{IP}{Internet Protocol}
\newacronym{tva}{TVA}{Tennessee Valley Authority}
\newacronym{vfd}{VFD}{Variable Frequency Drive}
\newacronym{onl}{ORNL}{Oak Ridge National Laboratory}
\newacronym{dos}{DDoS}{Distributed Denial of Service}
\newacronym{dnp}{DNP3}{Distributed Network Protocol 3}
\newacronym{dra}{DRA}{Dynamic Risk Assessment}
\newacronym{hmi}{HMI}{Human Machine Interface}
\newacronym{bol}{BOL}{Beginning-of-Life}
\newacronym{eol}{EOL}{End-of-Life}
\newacronym{mol}{MOL}{Middle-of-Life}
\newacronym{con}{IC}{Initial Condition}
\newacronym{dc}{DC}{Direct-Current}
\newacronym{ac}{AC}{Alternating-Current}
\newacronym{snl}{SNL}{Sandia National Laboratory}
\newacronym{cds}{CRDS}{Control Rod Drive System}
\newacronym{cdm}{CRDM}{Control Rod Drive Mechanism}
\newacronym{fma}{FMEA}{Failure Modes \& Effects Analysis}
\newacronym{rpn}{RPN}{Risk Priority Number}
\newacronym{scr}{SCR}{silicon controller rectifier}
\newacronym{hvc}{HVAC}{Heating, Ventilation \& Air Conditioning}
\newacronym{ttb}{TTB}{Time-to-Boil}
\newacronym{sis}{SIS}{Safety Instrumented System}
\newacronym{cis}{CISF}{Consolidated Interim Storage Facility}
\newacronym{iea}{IEA}{International Energy Agency}
\newacronym{snf}{SNF}{Spent Nuclear Fuel}
\newacronym{cic}{CISAC}{Center for Interantional Security and Cooperation}
\newacronym{usa}{US}{United states}
\newacronym{nwa}{NWPA}{Nuclear Waste Policy Act}
\newacronym{one}{NE}{Office of Nuclear Energy}
\newacronym{cgc}{CGC}{Common Ground Consortium}
\newacronym{cop}{CoPL}{Communities of Place}
\newacronym{cor}{CoPR}{Communities of Practice}
\newacronym{coh}{CoH}{Communities of Historical Experience}
\newacronym{ej}{EJ}{Environmental Justice}
\newacronym{ite}{ITEK}{Indigenous Traditional Technical Ecological Knowledge}
\newacronym{iae}{IAEA}{International Atomic Energy Agency}
\newacronym{rfi}{RFI}{Request for Information}
\newacronym{sta}{STAND}{Siting Tool for Advanced Nuclear Development}
\newacronym{ors}{OR-SAGE}{Oak Ridge Siting Analysis for Power Generation Expansion}
\newacronym{gis}{GIS}{Geographic Information System}
\newacronym{mtu}{MTU}{Metric Ton of Uranium}
\newacronym{mth}{MTHM}{Metric Ton of Heavy Metal}
\newacronym{isa}{ISA}{Idaho Settlement Agreement}
\newacronym{tru}{TRU}{Transuranic Waste}
\newacronym{cej}{CEJEST}{Climate and Economic Justice Screening Tool}
\newacronym{eja}{EJAtlas}{Global Atlas of Environmental Justice}
\newacronym{ejs}{EJScreen}{Environmental Justice Screening and Mapping Tool}
\newacronym{fpz}{FPTZ}{Fastest Path to Zero}
\newacronym{tra}{WebTRAGIS}{Web-Based Transportation Routing Analysis Geographic Information System}
\newacronym{lit}{LITE}{Land-area Identification, Tagging, and Exploration}
\newacronym{str}{START}{Stakeholder Tool for Assessing Radioactive Transportation}
\newacronym{nri}{NRIC}{National Reactor Innovation Center}
\newacronym{sec}{SEC}{Site Evaluation Criteria}
\newacronym{ceq}{CEQ}{Council on Environmental Quality}
\newacronym{epa}{EPA}{Environmental Protection Agency}
\newacronym{unf}{UNF-ST\&DARDS}{Used Nuclear Fuel-Storage, Transportation \& Disposal Analysis Resource and Data System}
\newacronym{gui}{GUI}{Graphical User Interface}
\newacronym{fem}{FEMA}{Federal Emergency Management Agency}
\newacronym{rap}{RAPT}{Resilience Analysis and Planning Tool}
\newacronym{gid}{GEOID}{Geographic Identifier}
\newacronym{icr}{I-CREWS}{Idaho Community-Engaged Resilience for Energy-Water Systems}
\newacronym{ew}{E-W}{Energy-Water}
\newacronym{ml}{ML}{Machine Learning}
\newacronym{set}{SETS}{Socio-Environmental Technological System}
\newacronym{rii}{RII}{Research Infrastructure Improvement}
\newacronym{aoo}{AOO}{Anticipated Operational Occurrence}
\newacronym{dbe}{DBE}{Design Basis Event}
\newacronym{bbe}{BDBE}{Beyond Design Basis Event}
\newacronym{fta}{FTA}{Fault Tree Analysis}
\newacronym{eta}{ETA}{Event Tree Analysis}
\newacronym{haz}{HAZOP}{Hazard and Operability Analysis}
\newacronym{onr}{ONR}{Office of Naval Research}
\newacronym{ui}{UI}{University of Idaho}
\newacronym{isu}{ISU}{Idaho State University}
\newacronym{bsu}{BSU}{Boise State University}
\newacronym{skp}{SKP}{Senior Key Personnel}
\newacronym{bpc}{BPC}{Broadening Participation in Computing}
\newacronym{epi}{EPI}{Energy Policy Institute}
\newacronym{ym}{YM}{Yucca Mountain}
\newacronym{nwf}{NWF}{Nuclear Waste Fund}
\newacronym{aea}{AEA}{Atomic Energy Act}
\newacronym{aec}{AEC}{Atomic Energy Commission}
\newacronym{un}{UN}{United Nations}
\newacronym{des}{DES}{Discrete Event Simulation}
\newacronym{nem}{NEIM}{Nuclear Engineering \& Industrial Management}
\newacronym{erd}{ERDA}{Energy Research and Development Administration}
\newacronym{rd}{R\&D}{Research \& Development}
\newacronym{nts}{NTS}{Nevada Test Site}
\newacronym{nrm}{NRMSAC}{Nevada Radioactive Materials Storage Advisory Committee}
\newacronym{tmi}{TMI}{Three Mile Island}
\newacronym{nwt}{NWTS}{National Waste Terminal Storage}
\newacronym{irg}{IRG}{Interagency Review Group on Nuclear Waste Management}
\newacronym{kwh}{kWh}{Kilowatt-Hour}
\newacronym{eis}{EIS}{Environmental Impact Statement}
%\newacronym{}{}{}
%Add \newacronym{nrs}{NRHES}{Nuclear Renewable Hybrid Energy System}
\newacronym{ahp}{AHP}{Analytical Hierarchy Process}
\newacronym{inl}{INL}{Idaho National Laboratory}
\newacronym{orl}{ORNL}{Oak Ridge National Laboratory}
\newacronym{anl}{ANL}{Argonne National Laboratory}
\newacronym{smr}{SMR}{Small Modular Reactor}
\newacronym{nus}{NuScale}{NuScale Power, LLC}
\newacronym{epri}{EPRI}{Electric Power Research Institute}
\newacronym{ci}{CI}{Consistency Index}
\newacronym{cr}{CR}{Consistency Ratio}
\newacronym{lwr}{LWR}{Light Water Reactor}
\newacronym{oer}{OER}{Online Educational Resource}
\newacronym{cps}{CPS}{Cyber-Physical Systems}
\newacronym{wsc}{WSC}{Western Services Corporation}
\newacronym{hsl}{HSSL}{Human System Simulation Laboratory}
\newacronym{roi}{ROI}{Return on Investment}
\newacronym{mwe}{MWe}{Megawatts-electric}
\newacronym{sca}{SCADA}{Supervisory Control and Data Acquisition}
\newacronym{udp}{UDP}{User Datagram Protocol}
\newacronym{plc}{PLC}{Programmable Logic Controller}
\newacronym{khp}{KHNP}{Korean Hydro \& Nuclear Power Co., Ltd}
\newacronym{jcp}{JCPOA}{Joint Comprehensive Plan of Action}
\newacronym{mim}{MITM}{Man in the Middle}
\newacronym{tcp}{TCP/IP}{Transmission Control Protocol/Internet Protocol}
\newacronym{pra}{PRA}{Probabilistic Risk Assessment}
\newacronym{cs}{CS}{Critical System}
\newacronym{loc}{LOCA}{Loss of Coolant Accident}
\newacronym{pha}{PHA}{Preliminary Hazards Analysis}
\newacronym{imu}{IMUNES}{Integrated Multiprotocol Network Emulator}
\newacronym{ccc}{CCC}{Computing Community Consortium}
\newacronym{neu}{NEUP}{Nuclear Energy University Program}
\newacronym{doe}{DOE}{United States Department of Energy}
\newacronym{nei}{NEI}{Nuclear Energy Institute}
\newacronym{nit}{NITRD}{Networking Information Technology Research \& Development Program}
\newacronym{rcs}{RCS}{Reactor Cooling System}
\newacronym{nic}{NIC}{Network Interface Card}
\newacronym{pc}{PCAP}{Packet Capture File}
\newacronym{csi}{CSIS}{Center for Strategic \& International Studies}
\newacronym{ste}{STEM}{Science, Technology, Engineering, and Mathematics}
\newacronym{pi}{PI}{Principal Investigator}
\newacronym{ans}{ANS}{American Nuclear Society}
\newacronym{sac}{SACNAS}{Society for the Advancement of Chicanos/Hispanics and Native Americans in Science}
\newacronym{ore}{ORED}{Office of Research and Economic Development}
\newacronym{rfd}{RFD}{Office of Research and Faculty Development}
\newacronym{nbe}{NSBE}{National Society of Black Engineers}
\newacronym{inp}{INPO}{Institute of Nuclear Power Operations}
\newacronym{nan}{NANTeL}{National Academy for Nuclear Training e-Learning}
\newacronym{cda}{CDA}{Critical Digital Assets}
\newacronym{nca}{NCA-CD}{National Center of Excellence in Cyber Defense}
\newacronym{ttx}{TTX}{Tabletop Exercise}
\newacronym{cie}{CIE}{Cyber-Informed Engineering}
\newacronym{aas}{AAS}{Adversary-As-A-Service}
\newacronym{iff}{UIIF}{Idaho Falls Center for Higher Education}
\newacronym{fy}{FY}{Fiscal Year}
\newacronym{eps}{EPSCoR}{Established Program to Stimulate Competitive Research}
\newacronym{nas}{NAS}{National Academies}
\newacronym{brc}{BRC}{Blue Ribbon Commission on America's Nuclear Future}
\newacronym{hlw}{HLW}{High-Level Radioactive Waste}
\newacronym{cbs}{CBS}{Consent-Based Siting}
\newacronym{wpp}{WIPP}{Waste Isolation Pilot Plant}
\newacronym{cfr}{CFR}{Code of Federal Regulations}
\newacronym{gao}{GAO}{Government Accountability Office}
\newacronym{npp}{NPP}{Nuclear Power Plant}
\newacronym{ump}{UAMPS}{Utah Associated Municipal Power Systems}
\newacronym{nrc}{NRC}{Nuclear Regulatory Commission}
\newacronym{nerc}{NERC}{North American Electric Reliability Corporation}
\newacronym{htse}{HTSE}{High Temperature Steam Electrolysis}
\newacronym{eia}{EIA}{U.S. Energy Information Administration}
\newacronym{lms}{LMS}{Learning Management System}
\newacronym{nsf}{NSF}{National Science Foundation}
\newacronym{cae}{CAES}{Center for Advanced Energy Studies}
\newacronym{pwr}{PWR}{Pressurized Water Reactor}
\newacronym{bwr}{BWR}{Boiling Water Reactor}
\newacronym{ic}{I\&C}{Instrumentation \& Controls}
\newacronym{ics}{ICS}{Industrial Control Systems}
\newacronym{ip}{IP}{Internet Protocol}
\newacronym{tva}{TVA}{Tennessee Valley Authority}
\newacronym{vfd}{VFD}{Variable Frequency Drive}
\newacronym{onl}{ORNL}{Oak Ridge National Laboratory}
\newacronym{dos}{DDoS}{Distributed Denial of Service}
\newacronym{dnp}{DNP3}{Distributed Network Protocol 3}
\newacronym{dra}{DRA}{Dynamic Risk Assessment}
\newacronym{hmi}{HMI}{Human Machine Interface}
\newacronym{bol}{BOL}{Beginning-of-Life}
\newacronym{eol}{EOL}{End-of-Life}
\newacronym{mol}{MOL}{Middle-of-Life}
\newacronym{con}{IC}{Initial Condition}
\newacronym{dc}{DC}{Direct-Current}
\newacronym{ac}{AC}{Alternating-Current}
\newacronym{snl}{SNL}{Sandia National Laboratory}
\newacronym{cds}{CRDS}{Control Rod Drive System}
\newacronym{cdm}{CRDM}{Control Rod Drive Mechanism}
\newacronym{fma}{FMEA}{Failure Modes \& Effects Analysis}
\newacronym{rpn}{RPN}{Risk Priority Number}
\newacronym{scr}{SCR}{silicon controller rectifier}
\newacronym{hvc}{HVAC}{Heating, Ventilation \& Air Conditioning}
\newacronym{ttb}{TTB}{Time-to-Boil}
\newacronym{sis}{SIS}{Safety Instrumented System}
\newacronym{cis}{CISF}{Consolidated Interim Storage Facility}
\newacronym{iea}{IEA}{International Energy Agency}
\newacronym{snf}{SNF}{Spent Nuclear Fuel}
\newacronym{cic}{CISAC}{Center for Interantional Security and Cooperation}
\newacronym{usa}{US}{United states}
\newacronym{nwa}{NWPA}{Nuclear Waste Policy Act}
\newacronym{one}{NE}{Office of Nuclear Energy}
\newacronym{cgc}{CGC}{Common Ground Consortium}
\newacronym{cop}{CoPL}{Communities of Place}
\newacronym{cor}{CoPR}{Communities of Practice}
\newacronym{coh}{CoH}{Communities of Historical Experience}
\newacronym{ej}{EJ}{Environmental Justice}
\newacronym{ite}{ITEK}{Indigenous Traditional Technical Ecological Knowledge}
\newacronym{iae}{IAEA}{International Atomic Energy Agency}
\newacronym{rfi}{RFI}{Request for Information}
\newacronym{sta}{STAND}{Siting Tool for Advanced Nuclear Development}
\newacronym{ors}{OR-SAGE}{Oak Ridge Siting Analysis for Power Generation Expansion}
\newacronym{gis}{GIS}{Geographic Information System}
\newacronym{mtu}{MTU}{Metric Ton of Uranium}
\newacronym{mth}{MTHM}{Metric Ton of Heavy Metal}
\newacronym{isa}{ISA}{Idaho Settlement Agreement}
\newacronym{tru}{TRU}{Transuranic Waste}
\newacronym{cej}{CEJEST}{Climate and Economic Justice Screening Tool}
\newacronym{eja}{EJAtlas}{Global Atlas of Environmental Justice}
\newacronym{ejs}{EJScreen}{Environmental Justice Screening and Mapping Tool}
\newacronym{fpz}{FPTZ}{Fastest Path to Zero}
\newacronym{tra}{WebTRAGIS}{Web-Based Transportation Routing Analysis Geographic Information System}
\newacronym{lit}{LITE}{Land-area Identification, Tagging, and Exploration}
\newacronym{str}{START}{Stakeholder Tool for Assessing Radioactive Transportation}
\newacronym{nri}{NRIC}{National Reactor Innovation Center}
\newacronym{sec}{SEC}{Site Evaluation Criteria}
\newacronym{ceq}{CEQ}{Council on Environmental Quality}
\newacronym{epa}{EPA}{Environmental Protection Agency}
\newacronym{unf}{UNF-ST\&DARDS}{Used Nuclear Fuel-Storage, Transportation \& Disposal Analysis Resource and Data System}
\newacronym{gui}{GUI}{Graphical User Interface}
\newacronym{fem}{FEMA}{Federal Emergency Management Agency}
\newacronym{rap}{RAPT}{Resilience Analysis and Planning Tool}
\newacronym{gid}{GEOID}{Geographic Identifier}
\newacronym{icr}{I-CREWS}{Idaho Community-Engaged Resilience for Energy-Water Systems}
\newacronym{ew}{E-W}{Energy-Water}
\newacronym{ml}{ML}{Machine Learning}
\newacronym{set}{SETS}{Socio-Environmental Technological System}
\newacronym{rii}{RII}{Research Infrastructure Improvement}
\newacronym{aoo}{AOO}{Anticipated Operational Occurrence}
\newacronym{dbe}{DBE}{Design Basis Event}
\newacronym{bbe}{BDBE}{Beyond Design Basis Event}
\newacronym{fta}{FTA}{Fault Tree Analysis}
\newacronym{eta}{ETA}{Event Tree Analysis}
\newacronym{haz}{HAZOP}{Hazard and Operability Analysis}
\newacronym{onr}{ONR}{Office of Naval Research}
\newacronym{ui}{UI}{University of Idaho}
\newacronym{isu}{ISU}{Idaho State University}
\newacronym{bsu}{BSU}{Boise State University}
\newacronym{skp}{SKP}{Senior Key Personnel}
\newacronym{bpc}{BPC}{Broadening Participation in Computing}
\newacronym{epi}{EPI}{Energy Policy Institute}
\newacronym{ym}{YM}{Yucca Mountain}
\newacronym{nwf}{NWF}{Nuclear Waste Fund}
\newacronym{aea}{AEA}{Atomic Energy Act}
\newacronym{aec}{AEC}{Atomic Energy Commission}
\newacronym{un}{UN}{United Nations}
\newacronym{des}{DES}{Discrete Event Simulation}
\newacronym{nem}{NEIM}{Nuclear Engineering \& Industrial Management}
\newacronym{erd}{ERDA}{Energy Research and Development Administration}
\newacronym{rd}{R\&D}{Research \& Development}
\newacronym{nts}{NTS}{Nevada Test Site}
\newacronym{nrm}{NRMSAC}{Nevada Radioactive Materials Storage Advisory Committee}
\newacronym{tmi}{TMI}{Three Mile Island}
\newacronym{nwt}{NWTS}{National Waste Terminal Storage}
\newacronym{irg}{IRG}{Interagency Review Group on Nuclear Waste Management}
\newacronym{kwh}{kWh}{Kilowatt-Hour}
\newacronym{eis}{EIS}{Environmental Impact Statement}
%\newacronym{}{}{}
%Add \input{$ACRONYM/acronyms} in the preamble, where $ACRONYM is the /path/to/file
 in the preamble, where $ACRONYM is the /path/to/file
 in the preamble, where $ACRONYM is the /path/to/file

%%%%%


\begin{document}


\nobibliography{borrelli-ui} %bib files
\bibliographystyle{resume} %custom bst file


\thispagestyle{empty} %start with page number 1 on second page


\begin{center}{\Large\textbf{CURRICULUM VITAE}\\\normalsize University of Idaho}\end{center}

\begin{spacing}{1.5}
\begin{longtable}{p{4.5 in}p{2.0 in}}
    \sj{name} Robert Angelo Borrelli 
    & \sj{date} \today
    \\
    \sj{rank or title} Associate Professor
    & 
    \\
    \sj{department} \acl{nem}
    &
    \\
    \sj{office location and campus zip} \acs{cae} 83401
    &
    \begin{enumerate}[topsep=0pt,itemsep=-0.75ex,partopsep=0ex,parsep=0ex,leftmargin=0.05in,label=(\arabic*)]
            \vspace{-0.235in}
%        \item[]\sj{phone} 208.533.8122
%        \item[]\sj{fax} 208.526.8255
        \item[]\sj{email} rborrelli@uidaho.edu
        \item[]\sj{web} \href{https://thedoctorrab.github.io/}{@TheDoctorRAB}
    \end{enumerate}
    \\
    \sj{date of first employment at ui} 2015.07.13
    &
    \\
    \sj{date of tenure} 2021.05.20
    &
    \\
    \sj{date of present rank or title} 2021.05.20
    &
\end{longtable}
\end{spacing}


\section*{Education Beyond High School}
    \begin{enumerate}[topsep=0pt,itemsep=0ex,partopsep=0ex,parsep=0ex,leftmargin=*,label=(\arabic*)]
        \item[]\textbf{Degrees}
        \item[]\textbf{Doctor of Philosophy --  Nuclear Engineering}\hfill{\textit{2006}}
        \item[]\textbf{University of California -- Berkeley}
        \item[]\textit{Radionuclide transport modeling with bentonite extrusion}
    \end{enumerate}
    \begin{enumerate}[topsep=7pt,itemsep=-0ex,partopsep=0ex,parsep=0ex,leftmargin=*,label=(\arabic*)]
        \item[]\textbf{Master of Science -- Civil \& Environmental Engineering}\hfill{\textit{1999}}
        \item[]\textbf{Worcester Polytechnic Institute}
        \item[]\textit{Characterization of background radiation in the environment}
    \end{enumerate}
    \begin{enumerate}[topsep=7pt,itemsep=-0ex,partopsep=0ex,parsep=0ex,leftmargin=*,label=(\arabic*)]
        \item[]\textbf{Bachelor of Science, Mechanical Engineering with high distinction}\hfill{\textit{1996}}
        \item[]\textbf{Worcester Polytechnic Institute}
        \item[]\textit{Capstone -- Real time PLC-based reactivity modeling by inverse point kinetics}
    \end{enumerate}


\section*{Experience}
    \begin{enumerate}[topsep=0pt,itemsep=0ex,partopsep=0ex,parsep=0ex,leftmargin=*,label=(\arabic*)]
        \item[]\textbf{Teaching, Extension, and Research Appointments}
        \item[]\textbf{\acl{ui} $\sq$ \acl{iff}}\hfill{\textit{2015--}}
        \item[]\textbf{Associate Professor $\sq$ \acl{nem}}\hfill{\textit{2021--}}
        \item[]\textbf{Assistant Professor $\sq$ \acl{nem}}\hfill{\textit{2015--21}}
        \item[]\textit{Affiliate Faculty -- \acl{ece} $\sq$ \acl{ui}}\hfill{\textit{2025--}}
        \item[]\textit{Coordinator -- Nuclear Technology Management Certificate}\hfill{\textit{2023--}}
        \item[]\textit{Affiliate -- \acl{epi} \acl{bsu}}\hfill{\textit{2019--}}
        \item[]\textit{Coordinator -- Nuclear Decommissioning \& Used Fuel Management Certificate}\hfill{\textit{2019--}}
        \item[]\textit{State of Idaho Professional Engineer -- Faculty Restricted }\hfill{\textit{2019--}}
        \item[]\textit{Coordinator -- Nuclear Criticality Safety Certificate}\hfill{\textit{2015--}}
    \end{enumerate}
    \begin{enumerate}[topsep=7pt,itemsep=-0ex,partopsep=0ex,parsep=0ex,leftmargin=*,label=(\arabic*)]
        \item[]\textbf{Postdoctorate Researcher}\hfill{\textit{2009-12}}
        \item[]\textbf{University of California-Berkeley $\sq$ Nuclear Engineering}
        \item[]\textit{Safeguardability methodology for remotely-handled nuclear materials facilities}
    \end{enumerate}
    \begin{enumerate}[topsep=7pt,itemsep=-0ex,partopsep=0ex,parsep=0ex,leftmargin=*,label=(\arabic*)]
        \item[]\textbf{Research Associate}\hfill{\textit{2007--09}}
        \item[]\textbf{University of Tokyo $\sq$ Nuclear Engineering \& Management}
        \item[]\textit{Mathematical modeling for mass transport in the engineered barrier system of a high-level nuclear waste repository}
    \end{enumerate}
    \begin{enumerate}[topsep=7pt,itemsep=-0ex,partopsep=0ex,parsep=0ex,leftmargin=*,label=(\arabic*)]
        \item[]\textbf{Postdoctorate Researcher}\hfill{\textit{2007}}
        \item[]\textbf{University of California-Berkeley $\sq$ Nuclear Engineering}
        \item[]\textit{Mathematical modeling for mass transport in the engineered barrier system of a high-level nuclear waste repository}
    \end{enumerate}
    \begin{enumerate}[topsep=7pt,itemsep=-0ex,partopsep=0ex,parsep=0ex,leftmargin=*,label=(\arabic*)]
        \item[]\textbf{Doctoral Candidate}\hfill{\textit{2005-06}}
        \item[]\textbf{University of California-Berkeley $\sq$ Nuclear Engineering}
        \item[]\textit{Derived a two-phase, mass transport model for radionuclides in a porous medium with bentonite extrusion model in a planar fracture to assess the potential to confine radionuclides.}
    \end{enumerate}
    \begin{enumerate}[topsep=7pt,itemsep=-0ex,partopsep=0ex,parsep=0ex,leftmargin=*,label=(\arabic*)]
        \item[]\textbf{Intern, Earth Sciences Division}\hfill{\textit{2001--02}}
        \item[]\textbf{Lawrence Berkeley National Laboratory}
        \item[]\textit{Data analysis for chaotic modeling of water flow in the unsaturated zone}
    \end{enumerate}


\section*{Experience}
    \begin{enumerate}[topsep=0pt,itemsep=0ex,partopsep=0ex,parsep=0ex,leftmargin=*,label=(\arabic*)]
        \item[]\textbf{Non-Academic Employment including Armed Forces}
        \item[]\textbf{Assistant Radiation Safety Officer}\hfill{\textit{1996-99}}
        \item[]\textbf{Leslie C. Wilbur Nuclear Reactor Facility}
        \item[]\textbf{Worcester Polytechnic Institute}
        \item[]\textit{Emergency response for incidents involving radioactive material; Records maintenance of radiation exposures to campus personnel, etc.}
    \end{enumerate}
    \begin{enumerate}[topsep=7pt,itemsep=-0ex,partopsep=0ex,parsep=0ex,leftmargin=*,label=(\arabic*)]
        \item[]\textbf{Senior Nuclear Reactor Operator \#70145}\hfill{\textit{1994-99}}
        \item[]\textbf{Leslie C. Wilbur Nuclear Reactor Facility}
        \item[]\textbf{Worcester Polytechnic Institute}
        \item[]\textit{Emergency response for incidents involving radioactive material; Records maintenance of radiation exposures to campus personnel, etc.}
    \end{enumerate}
    \begin{enumerate}[topsep=7pt,itemsep=-0ex,partopsep=0ex,parsep=0ex,leftmargin=*,label=(\arabic*)]
        \item[]\textbf{Major Qualifying Project}\hfill{\textit{1995--96}}
        \item[]\textbf{Leslie C. Wilbur Nuclear Reactor Facility}
        \item[]\textbf{Worcester Polytechnic Institute}
        \item[]\textit{The Major Qualifying Project synthesizes previous undergraduate studies to solve problems in the major field and communicate results. For this project, a logic algorithm was developed for reactivity derived from point kinetics equations and a \acl{plc} was modified for real time data collection during nuclear reactor operation for use with experimental coursework and maintenance procedures.}
    \end{enumerate}


\section*{Teaching Accomplishments}
    \begin{enumerate}[topsep=0pt,itemsep=0ex,partopsep=0ex,parsep=0ex,leftmargin=*,label=(\arabic*)]
        \item[]\textbf{Areas of Specialization} Neutronics, risk assessment, pyroprocessing, nuclear fuel cycle analysis, safeguards, security, nuclear cybersecurity, nuclear integrated hybrid energy systems, regulatory analysis, energy policy
    \end{enumerate}
    \begin{enumerate}[topsep=7pt,itemsep=-0ex,partopsep=0ex,parsep=0ex,leftmargin=*,label=(\arabic*)]
        \item[]\textbf{University of Idaho $\sq$ Idaho Falls Center for Higher Education}
        \item[]\textbf{\acl{nem}}
        \item[]\textbf{Used Fuel Management \& Nuclear Power Plant Decommissioning Certificate}\hfill{\textit{Offered variably as needed}}
        \item[]\textit{NE514: Nuclear Safety}
        \item[]The focus of this course is on the approach to nuclear safety for the commercial nuclear industry. Historical events serve as a basis for learning to investigate how nuclear energy policy has affected approaches to safety developed in the United States. Additionally, how safety approaches affect the nuclear industry will be covered. Several different classes of nuclear facilities will be studied. Relevant current events will be highlighted.
    \end{enumerate}
    \begin{enumerate}[topsep=7pt,itemsep=-0ex,partopsep=0ex,parsep=0ex,leftmargin=*,label=(\arabic*)]
        \item[]\textit{NE516: Nuclear Rules \& Regulations}
        \item[]The focus of this course is on the rules and regulations that govern the commercial nuclear industry. Historical events serve as a basis for learning to investigate how nuclear energy policy has developed in the United States. This includes not only current regulations but important development that led to major policy changes. Additionally, how current regulations affect the nuclear industry will be covered. Facilities studied will include research, commercial, and government. Relevant current events will be highlighted.
    \end{enumerate}
    \begin{enumerate}[topsep=7pt,itemsep=-0ex,partopsep=0ex,parsep=0ex,leftmargin=*,label=(\arabic*)]
        \item[]\textit{NE527: Nuclear Materials, Transportation, Storage}\hfill{\textit{in development}}
        \item[]The focus of this course is on the scope and sequencing of activities necessary to comply with the rules and regulations that govern storage, transport, and disposal of nuclear materials. The course is mainly focused on spent nuclear fuel management, Department of Energy high-level waste management, and NRC regulated activities. The course is based on United States nuclear facilities and nuclear power plants.
    \end{enumerate}
    \begin{enumerate}[topsep=7pt,itemsep=-0ex,partopsep=0ex,parsep=0ex,leftmargin=*,label=(\arabic*)]
        \item[]\textit{NE587: Nuclear Facility Decommissioning}
        \item[]The focus of this course is on the scope and sequencing of activities necessary to comply with the rules and regulations that govern decommissioning of nuclear facilities. The course is mainly focused on the NRC regulated activities, as these are both stricter and more regimented when compared to the DOE/DOD regulated activities. However, most of the rules apply to both. The course is based on D\&D experience in the US, both for facilities which have successfully completed the process and for facilities currently in the process.
    \end{enumerate}
    \begin{enumerate}[topsep=15pt,itemsep=-0ex,partopsep=0ex,parsep=0ex,leftmargin=*,label=(\arabic*)]
        \item[]\textbf{Courses Taught}
        \item[]\textbf{\acl{ui} $\sq$ \acl{iff}}
        \item[]\textbf{\acl{nem}}
        \item[]\textit{NE587: Nuclear Facility Decommissioning}\hfill{\textit{Fall 2025}}
        \item[]\hfill{\textit{supervising instructor}}
        \item[]The focus of this course is on the scope and sequencing of activities necessary to comply with the rules and regulations that govern decommissioning of nuclear facilities. The course is mainly focused on the NRC regulated activities, as these are both stricter and more regimented when compared to the DOE/DOD regulated activities. However, most of the rules apply to both. The course is based on D\&D experience in the US, both for facilities which have successfully completed the process and for facilities currently in the process.
    \end{enumerate}
    \begin{enumerate}[topsep=7pt,itemsep=-0ex,partopsep=0ex,parsep=0ex,leftmargin=*,label=(\arabic*)]
        \item[]\textit{NE585: Nuclear Fuel Cycle Analysis}\hfill{\textit{Fall 2023--; 2017}}
        \item[]This course presents the nuclear fuel cycle can as an holistic system with components related in many complex ways. This course focuses on systems analysis of components that comprise the nuclear fuel cycle to understand the contemporary challenges facing nuclear energy. Topics include reactor design, critical size, reactor statics and dynamics, advanced reactor design, and back-end management; siting, fuel management, interim storage, repository design. Students will also gain facility with MCNP. 
    \end{enumerate}
    \begin{enumerate}[topsep=7pt,itemsep=-0ex,partopsep=0ex,parsep=0ex,leftmargin=*,label=(\arabic*)]
        \item[]\textit{NE529: Risk Assessment}\hfill{\textit{Spring 2021--; 2016--19}}
        \item[]This course is designed to provide students with an understanding of how to perform a comprehensive risk assessment applicable to a wide variety of engineering problems in many different disciplines. The course will focus on failure mode and effect analysis, fault tree analysis, probabilistic risk analysis, and human reliability analysis. The course will also cover fundamental probability and statistics content.
    \end{enumerate}
    \begin{enumerate}[topsep=7pt,itemsep=-0ex,partopsep=0ex,parsep=0ex,leftmargin=*,label=(\arabic*)]
        \item[]\textit{NE544: Reactor Analysis -- Statics and Kinetics}\hfill{\textit{Spring 2022 --}}
        \item[]\hfill{\textit{supervising instructor}}
        \item[]The purpose of this course is to study nuclear theory in the context of nuclear reactor engineering; concepts relating to the design and operation of nuclear reactors will be discussed. Content includes development of mathematical expressions describing the relevant nuclear processes as well as their physical implications. This course will involve the application of several common analytical tools used for the design and evaluation of nuclear systems.
    \end{enumerate}
    \begin{enumerate}[topsep=7pt,itemsep=-0ex,partopsep=0ex,parsep=0ex,leftmargin=*,label=(\arabic*)]
        \item[]\textit{NE450: Principles of Nuclear Engineering}\hfill{\textit{Fall 2015--22}}
        \item[]In this course, an overview of fundamental nuclear engineering principles and how these are practically applied to contemporary, nuclear engineering problems will be presented.  The topics covered in this course include: nuclear physics and reactions, materials science, radiation protection, energy production, fuel cycle analysis, advanced reactor design, fusion, nonproliferation, back-end management, and risk assessment and safety.  Throughout the course, the ethical considerations with regards to engineering problems within these fields will also be discussed.
    \end{enumerate}
    \begin{enumerate}[topsep=7pt,itemsep=-0ex,partopsep=0ex,parsep=0ex,leftmargin=*,label=(\arabic*)]
        \item[]\textit{NE502: Python \acs{mcp} Development for Molten Fuel Salt Handling}\hfill{\textit{Spring 2025}}
        \item[]In this course, a series of Python Scripts are developed to construct \acs{mcp} input decks that will be applied to establish criticality safety fuel handling controls for molten fuel salt.
    \end{enumerate}
    \begin{enumerate}[topsep=7pt,itemsep=-0ex,partopsep=0ex,parsep=0ex,leftmargin=*,label=(\arabic*)]
        \item[]\textit{NE502: Nuclear Power Plant Decommissioning Education}\hfill{\textit{Fall 2023}}
        \item[]As variable energy sources increasingly penetrate the United States energy market, the economics of nuclear energy has weakened. This has led to planned plant closures. Under NRC regulations, these plants must be decommissioned; i.e., safe removal from service and reduction of radioactivity to permissible levels in order to release the property. These activities will require specialized training. The market for decommissioning in North America was valued at \$1.92B in 2019, and it is projected to increase to \$3.35B in 2027. A significant, skilled, and technically proficient workforce will be needed to complete these activities safely and within regulations. Currently, there are no programs in the US specifically focused on decommissioning workforce training. This course will develop a nuclear power plant decommissioning asynchronous, online graduate course as part of the University of Idaho Nuclear Power Plant Decommissioning Graduate Certificate.
    \end{enumerate}
    \begin{enumerate}[topsep=7pt,itemsep=-0ex,partopsep=0ex,parsep=0ex,leftmargin=*,label=(\arabic*)]
        \item[]\textit{NE502: Computation and modeling of nuclear systems}\hfill{\textit{Fall 2020}}
        \item[]Using published nuclear computational modeling references, this course will examine prior efforts to simplify the neutron transport and depletion calculations for advanced reactor designs, such as high-temperature pebble-bed reactors. These include efforts to model heterogeneous stochastic media, such as TRISO fuel particles dispersed in a graphite matrix. Deterministic neutron transport in stochastic media has resulted in several new methods developed in order to address the challenges associated with modeling them, however, comparison of the proposed and implemented methods has been largely absent. Other challenges facing the modeling of pebble-bed reactors include the depletion calculations for individual pebbles. Because of the flow of pebbles through the core, and the recirculation of said pebbles in most modern designs, the depletion of individual pebbles can affect the criticality of the system as well as the temperature profile of individual pebbles, which was demonstrated to have significant safety implications based on the operation of the German AVR pebble-bed system. To these ends, this course will identify relevant literature regarding these topics and the feasibility of implementing these methods in modern nuclear modeling tools, such as the Griffin tool in development at Idaho National Laboratory.
    \end{enumerate}
    \begin{enumerate}[topsep=7pt,itemsep=-0ex,partopsep=0ex,parsep=0ex,leftmargin=*,label=(\arabic*)]
        \item[]\textit{NE502: Heterogeneity comparisons of intermediate enrichment uranium in critical systems}\hfill{\textit{Fall 2020}}
        \item[]The ICSBEP handbook contains descriptions and evaluations of critical experiments conducted in facilties throughout the world. The reach of the handbook stretches to the beginning of the history of nuclear engineering. As part of that history, a number of critical experiments have been conducted using arrays of fissile material in the intermediate enrichment range, which is defined for the purposes of this course as ten to forty percent (10\%-40\%). In those evaluations, a variety of methods have been discussed and described which convert a system of known or unknown degree of heterogeneity into a simplified homogeneous model, which is a traditional method for examining the critical characteristics of multiplying systems. The specific methodology for this conversion varies between evaluations and is explained in some cases, where other cases are less descriptive. It is of interest to the field of nuclear criticality safety to improve the understanding of the effect of heterogeneity on the critical characteristics of multiplying fissile systems, particularly in the range of intermediate enrichment. It is of further interest to the field that a methodology be established which can be used to evaluate these types of systems for practical nuclear criticality safety applications, especially the development of nuclear criticality safety limits.
    \end{enumerate}
    \begin{enumerate}[topsep=7pt,itemsep=-0ex,partopsep=0ex,parsep=0ex,leftmargin=*,label=(\arabic*)]
        \item[]\textit{NE502: Computation of spherical critical volume for nuclear criticality safety applications} \hfill{\textit{Fall 2020}}
        \item[]The nuclear safety guide, Critical Dimensions of Systems Containing U-235, Pu-239, and U-233 contains a graph of minimum critical volumes as a function of U-235 enrichment in homogeneous and heterogeneous hydrogen-moderated systems. This graph includes a wide extrapolation between the data for ten percent enrichment and ninety-three percent enrichment. A similar graph appears in Anomalies of Nuclear Criticality without the extrapolation. This graph shows a clear transition at approximately thirty-five percent enrichment between two different types of systems. It is clear that there is a difference between the extrapolated minimum spherical critical volumes and the calculated critical volumes. As part of the development of nuclear criticality safety methodology for intermediate-enrichment uranium systems, it is necessary to more fully evaluate and explain the nature of the transition point. This work will attempt to re-create and expand on the data used to produce this graph and more fully evaluate the nature of the minimum critical volume for intermediate-enrichment systems. The secondary objective will be the proposal of nuclear criticality safety rules which could be applied to these data to ensure that fissionable material operations in this range of enrichment can be conducted safely.
    \end{enumerate}
    \begin{enumerate}[topsep=7pt,itemsep=-0ex,partopsep=0ex,parsep=0ex,leftmargin=*,label=(\arabic*)]
        \item[]\textit{NE502: Subcritical multiplication} \hfill{\textit{Fall 2020}}
        \item[]Using published fundamental nuclear data for thermal neutrons, this course will examine the relationship between processes of fission and capture and develop models for reactivity worth as a function of thermal fission and capture cross-sections and local thermal flux. This course will include evaluation of tests in zero-power reactors. The Advanced Test Reactor (ATR) physics analyses rely heavily on measurement results from the Advanced Test Reactor Critical Facility (ATRC). These results are easily delivered in terms of reactivity worth, but the results are almost never directly applicable to ATR due to differences in thermal neutron flux. Reactivity worth is not a fundamental property of a material and indeed is not a characteristic property of a given piece of reactor hardware if the hardware is to be irradiated in a new neutron environment. However, translating reactivity worth of a piece of hardware in ATRC to a pair of macroscopic cross-sections essentially characterizes the hardware with constants that will be applicable in any subsequent location in ATR or ATRC. Therefore, reactivity worth in a new irradiation position can be predicted by calculation.
    \end{enumerate}
    \begin{enumerate}[topsep=7pt,itemsep=-0ex,partopsep=0ex,parsep=0ex,leftmargin=*,label=(\arabic*)]
        \item[]\textit{NE502: Historical examinations of heterogeneity in nuclear criticality safety} \hfill{\textit{Spring 2020}}
        \item[]Using published nuclear criticality safety references, this course will conduct an examination of the effect of heterogeneity with respect to nuclear criticality safety. As part of the history of nuclear criticality safety, a number of critical experiments have been conducted using arrays of fissile material in an attempt to quantify the effect of heterogeneity in the determination of safe handling limits. In these attempts, a number of low-enriched experiments were examined and converted in their critical dimensions to establish a useful baseline from which other limits could be extrapolated. It is of interest to the field of nuclear criticality safety to improve the understanding of the effect of heterogeneity on the critical characteristics of multiplying fissile systems, particularly in the range of intermediate enrichment. It is of further interest to the field that a methodology be established which can be used to evaluate these types of systems for practical nuclear criticality safety applications, especially the development of nuclear criticality safety limits. To that end, this course will identify relevant data in the literature used to generate historical experiments and apply MCNP to model these experiments in order to understand and evaluate assumptions and restrictions raised in the experiments.
    \end{enumerate}
    \begin{enumerate}[topsep=7pt,itemsep=-0ex,partopsep=0ex,parsep=0ex,leftmargin=*,label=(\arabic*)]
        \item[]\textit{NE502: Heterogeneity comparisons of intermediate enrichment uranium in critical systems}\hfill{\textit{Spring 2020}}
        \item[]Using the International Criticality Safety Benchmark Evaluation Project (ICSBEP) handbook of evaluated critical experiments, conduct an examination of the effect of heterogeneity in intermediate-enrichment uranium systems. The ICSBEP handbook contains descriptions and evaluations of critical experiments conducted in facilities throughout the world. The reach of the handbook stretches to the beginning of the history of nuclear engineering. In those evaluations, a variety of methods have been discussed and described which convert a system of known or unknown degree of heterogeneity into a simplified homogeneous model, which is a traditional method for examining the critical characteristics of multiplying systems. The specific methodology for this conversion varies between evaluations and is explained in some cases, where other cases are less descriptive. It is of interest to the field of nuclear criticality safety to improve the understanding of the effect of heterogeneity on the critical characteristics of multiplying fissile systems, particularly in the range of intermediate enrichment. It is of further interest to the field that a methodology be established which can be used to evaluate these types of systems for practical nuclear criticality safety applications, especially the development of nuclear criticality safety limits. This course will identify and evaluate relevant benchmark experiments for heterogeneity effects using MCNP to establish new baseline models from the benchmark handbook.
    \end{enumerate}
    \begin{enumerate}[topsep=7pt,itemsep=-0ex,partopsep=0ex,parsep=0ex,leftmargin=*,label=(\arabic*)]
        \item[]\textit{NE502: Nuclear integrated energy systems} \hfill{\textit{Spring 2020}}
        \item[]Nuclear renewable hybrid energy systems enable a nuclear reactor to load follow with a renewable energy source. These must be designed to distribute energy dynamically by supplying electricity to the grid while using either thermal or electrical energy for industrial applications. This takes advantage of the flexible distribution of electricity or heat to maximize profit. The industrial process serves as a load sink for the excess heat or electricity produced by the nuclear reactor. Economic challenges to the current United States nuclear light water reactor (LWR) fleet have led to early plant closures. While LWRs primarily deliver baseload electricity, there is no reason why nuclear energy produced by these reactors cannot be used to provide energy to a range of industrial applications. This directed study course will identify feasible systems, products, and commodities that could be produced by existing nuclear plants. As part of this, cost and potential profitability will be analyzed within the context of market structures and grid reliability.
    \end{enumerate}
    \begin{enumerate}[topsep=7pt,itemsep=-0ex,partopsep=0ex,parsep=0ex,leftmargin=*,label=(\arabic*)]
        \item[]\textit{NE527: Nuclear material storage, transport, disposal} \hfill{\textit{Spring 2020}}
        \item[]There is a wide range of nuclear materials that are stored, transported and disposed of each day. The materials include medical radioisotopes, new fuel pellets, used fuel, and industrial radioisotopes. This course will cover the regulations that govern nuclear material storage, transportation and disposal, as well as the engineering requirements and practical aspects of handling these materials.
    \end{enumerate}
    \begin{enumerate}[topsep=7pt,itemsep=-0ex,partopsep=0ex,parsep=0ex,leftmargin=*,label=(\arabic*)]
        \item[]\textit{NE535: Nuclear Criticality Safety I}\hfill{\textit{Spring 2022; 2020}}
        \item[]This course applies uses the foundation of applied nuclear physics to develop and explain the international and domestic rules and practices that are used to prevent inadvertent criticality in fuel cycle applications such as used fuel storage and processing.
    \end{enumerate}
    \begin{enumerate}[topsep=9pt,itemsep=-0ex,partopsep=0ex,parsep=0ex,leftmargin=*,label=(\arabic*)]
        \item[]\textbf{Diablo Valley College}
        \item[]\textbf{Architecture and Engineering}
        \item[]\textbf{Adjunct Professor}
        \item[]\textit{ENGIN110: Introduction to Engineering}\hfill{\textit{2013--15}}
        \item[]This course introduces students to fundamental engineering principles. Students learn how these are applied to contemporary engineering problems through laboratory exercises, homework assignments, design projects, interviews with professional engineers, and field trips to engineering companies. Topics include: materials science, risk assessment and safety, critical problem-solving, engineering analysis, engineering design processes, project development, engineering software, and presentation tools. The role of the engineer in society, professionalism, and engineering ethics are major themes. The emphasis is on creative problem-solving, teamwork, and effective communication, both in presentation and writing.
    \end{enumerate}
    \begin{enumerate}[topsep=9pt,itemsep=-0ex,partopsep=0ex,parsep=0ex,leftmargin=*,label=(\arabic*)]
        \item[]\textbf{University of California --Berkeley}
        \item[]\textbf{Nuclear Engineering}
        \item[]\textbf{Instructor}
        \item[]\textit{NE375: Teaching Techniques in Nuclear Engineering}\hfill{\textit{2006; 2010--11}}
        \item[]This course acquaints \acfp{gsi} with teaching techniques for courses in the Department of Nuclear Engineering. The \acs{gsi} will have several duties far beyond grading assignments and/or examinations: conducting discussion sessions, review lectures, or laboratory experiments. The \acs{gsi}, therefore, needs to develop the appropriate tools to use when facing these pedagogical challenges. Three students from the 2010 course received the Outstanding Graduate Student Instructor Award given by the UC-Berkeley \acl{gsi} Teaching \& Resource Center.
    \end{enumerate}
    \begin{enumerate}[topsep=7pt,itemsep=-0ex,partopsep=0ex,parsep=0ex,leftmargin=*,label=(\arabic*)]
        \item[]\textbf{\acl{gsi}}
        \item[]\textit{E124: Ethics and the Impact of Technology on Society}\hfill{\textit{2004--06}}
        \item[]Conducted multiple discussion sections on a weekly basis and review lectures
        \item[]Supervised research projects based on current, ethical and scientific issues 
        \item[]Assessed individual student presentations, projects, and overall course grading
    \end{enumerate}
    \begin{enumerate}[topsep=7pt,itemsep=-0ex,partopsep=0ex,parsep=0ex,leftmargin=*,label=(\arabic*)]
        \item[]\textit{IDS110: Introduction to Computing}\hfill{\textit{2004}}
        \item[]Conducted multiple laboratory sessions on a weekly basis
        \item[]Supervised undergraduate research projects focused on web based education
        \item[]Assessed laboratory assignments and project grading
    \end{enumerate}
    \begin{enumerate}[topsep=7pt,itemsep=-0ex,partopsep=0ex,parsep=0ex,leftmargin=*,label=(\arabic*)]
        \item[]\textit{NE92: Issues in Nuclear Science and Engineering}\hfill{\textit{2000; 2002}}
        \item[]Conducted multiple laboratory sessions on a weekly basis
        \item[]Supervised undergraduate research projects focused on web based education
        \item[]Assessed laboratory assignments and project grading
    \end{enumerate}
    \begin{enumerate}[topsep=7pt,itemsep=-0ex,partopsep=0ex,parsep=0ex,leftmargin=*,label=(\arabic*)]
        \item[]\textit{NE275: Principles and Methods of Risk Analysis}\hfill{\textit{2001}}
        \item[]This graduate course requires a deeper understanding of the subject matter, due to the student body. The course was one of three in the curriculum with the highest credit load. The main responsibility in this was to advise and grade semester projects and presentations based on risk assessments of engineering systems.
    \end{enumerate}
    \begin{enumerate}[topsep=7pt,itemsep=-0ex,partopsep=0ex,parsep=0ex,leftmargin=*,label=(\arabic*)]
        \item[]\textbf{Reader}
        \item[]\textit{NE150: Introduction to Nuclear Reactor Theory}\hfill{\textit{2003}}
        \item[]\textit{NE104: Radiation Detection and Nuclear Instrumentation Laboratory}\hfill{\textit{2002}}
        \item[]\textit{NE107: Introduction to Imaging}\hfill{\textit{2001}}
        \item[]\textit{NE120: Nuclear Materials}\hfill{\textit{2000}}
        \item[]Supervised laboratory sessions
        \item[]Assessed examinations, homework assignments, laboratory reports, final grades
        \item[]Conducted review lectures
    \end{enumerate}
    \begin{enumerate}[topsep=9pt,itemsep=-0ex,partopsep=0ex,parsep=0ex,leftmargin=*,label=(\arabic*)]
        \item[]\textbf{University of Tokyo}
        \item[]\textbf{Nuclear Engineering \& Management}
        \item[]\textbf{Part Time Lecturer}
        \item[]\textit{Technical English for Scientists}\hfill{\textit{2007--08}}
        \item[]This course provided the opportunity for non-native English speaking students to develop technical communication skills; i.e., presenting scientific and technical material to an informed audience at an international conference. In this course, the ‘assertion evidence design’ concept for technical presentation of scientific topics was applied to student research interests. Transmutable skills focused on the professional communication of scientific research in various public speaking formats and a comfortable familiarity with the English language to establish a stronger foundation for technical writing.
    \end{enumerate}


\section*{Students Advised}
    \renewcommand*{\arraystretch}{1.25}
    \begin{longtable}{m{1.25 in}L{1.25 in}L{2.50 in}p{0.30in}p{0.05in}}
        \multicolumn{4}{l}{\textbf{\acl{ui}}\hfill{\textit{Affiliation is of where majority degree research was conducted}}}\\
        \multicolumn{4}{l}{\textbf{Major Professor}}\\
        \multicolumn{4}{l}{\textit{Current}}\\
        Dylan Ohrt 
        & \acl{inl}
        & Nuclear materials safeguards 
        & \acs{phd}
        &
        \\
        J. Seth Dustin 
        & Oklo
        & Waste processing
        & \acs{phd}
        &
        \\
        Nathan Manwaring
        & \acl{inl}
        & Criticality safety
        & \acs{phd}
        &
        \\
        Trevin Lasley
        & \acl{nrf}
        & Criticality safety
        & \acs{phd}
        &
        \\
        David Haar 
        & Self-employed
        & \acl{dnd}
        & \acs{phd}
        &
        \\
        Kevin Haar 
        & \acl{wpp}
        & 
        & \acs{ms}
        &
        \\
        Ethan Bauer
        & \acl{inl}
        & Detector design
        & \acs{ms}
        &
        \\
        \\
        \multicolumn{4}{l}{\textit{Graduates}}\\
        Olin Calvin
        & \acl{ui}
        & Depletion chain simplification with pseudo-nuclides to model decay effects
        & \acs{phd}
        & \textit{2023}
        \\
        Joseph Christensen
        & TerraPower
        & Advancements in the Evaluation of Heterogeneity for Nuclear Criticality Safety in High-Assay Low-Enriched Uranium Systems
        & \acs{phd}
        & \textit{2023}
        \\
        Teyen Widdicombe
        & \acl{ui}
        & Investigation of Interactions Between Radiations from Dragonfly's MMRTG and Titan's Environment
        & \acs{phd}
        & \textit{2022}
        \\
        Kelley M. Verner
        & \acl{ui}
        & Irradiation induced phase change in low enriched uranium-molybdenum fuel as it relates to microstructure
        & \acs{phd}
        & \textit{2021}
        \\
        J. Seth Dustin
        & \acl{ui}
        & High level examination of Am-241 as an alternative Fuel Source in radioisotope thermoelectric generators
        & \acs{ms}
        & \textit{2020}
        \\
        John Peterson
        & \acl{ui}
        & Molten salt reactor neutronics design
        & \acs{ms}
        & \textit{2019}
        \\
        Jieun Lee
        & \acl{ui}
        & Risk-informed safeguards of pyroprocessing for advanced nuclear fuel concepts
        & \acs{ms}
        & \textit{2018}
        \\
        Emma Redfoot
        & \acl{ui}
        & Allocating heat and electricity in a nuclear renewable hybrid energy system coupled with a water purification system
        & \acs{ms}
        & \textit{2018}
        \\
        Malachi Tolman
        & \acl{ui}
        & \acs{inl} BISON code documentation
        & \acs{ms}
        & \textit{2017}
        \\
        Jonathon Wheelwright
        & \acl{inl}
        & Microreactor transport
        & MEng
        & \textit{2023}
        \\
        Trevor MacLean
        & \acl{inl}
        & Cybersecurity modeling of non-critical nuclear power plant instrumentation
        & MEng
        & \textit{2018}
    \end{longtable}


    \renewcommand*{\arraystretch}{1.25}
    \begin{longtable}{m{1.25 in}L{1.25 in}L{2.50 in}p{0.30in}p{0.05in}}
        \multicolumn{4}{l}{\textbf{\acl{ui}}}\\
        \multicolumn{4}{l}{\textbf{Committee Member}}\\
        \multicolumn{4}{l}{\textit{Graduates}}\\
        Jonathan Tacke 
        & \acl{ui}
        & Frequency Regulation by Way of a Variable Moment of Inertia
        & \acs{phd}
        & \textit{2024}
        \\
        Ryan C. Hruska        
        & \acl{inl}
        & A Functional All-Hazard Approach to Critical Infrastructure Dependency Analysis  
        & \acs{phd}
        & \textit{2023}
        \\
        David Kamerman
        & \acl{inl}
        & On the Role of Bulk Hydrides and Hydride Rims in Causing Low Temperature Ruptures of Zircaloy-4 Cladding Tubes in Transient Reactor Tests
        & \acs{phd}
        & \textit{2023}
        \\
        James Richards 
        & Ultra Safe Nuclear Corp.
        & Techno-Economic Analysis of Nuclear Integrated Energy Systems for Water Desalination and Hydrogen Production
        & \acs{phd}
        & \textit{2023}
        \\
        John Carter 
        & \acl{inl}
        & Core design of molten salt nuclear battery
        & \acs{phd}
        & \textit{2022}
        \\
        Jacob Benjamin 
        & Dragos, Inc.
        & Bounding cyber in design basis threat 
        & \acs{phd}
        & \textit{2020}
        \\
        Gabriel Lewis
        & \acl{ui}
        & Redox measurement and Corrosion testing in molten FLiNaK salts
        & \acs{ms}
        & \textit{2025}
        \\
        Sam J. Root
        & \acl{ui}
        & Dynamic System Modeling \& PID Controller Design for a Molten Salt Microreactor
        & \acs{ms}
        & \textit{2024}
        \\
        Stefan Abbot
        & \acl{ui}
        & Electrochemical Analysis of Molybdenum, TZM, and Molybdenum-Lanthanum ODS in Aqueous Solution
        & \acs{ms}
        & \textit{2024}
        \\
        Kendall Bean
        & \acl{ui}
        & Using static VAR Compensators to Simultaneously Regulate Power System Voltage and Frequency
        & \acs{ms}
        & \textit{2022}
        \\
        Trevin Lasley 
        & \acl{ui}
        & Criticality safety on the molten salt nuclear battery
        & \acs{ms}
        & \textit{2021}
        \\
        Joshua Young
        & \acl{ui}
        & Analysis of a Dump Heat Exchanger For The Versatile Test Reactor’s Secondary Loop
        & \acs{ms}
        & \textit{2021}
        \\
        Jonathan Tacke 
        & \acl{ui}
        & Design of an automatic voltage regulator with limited plant information
        & \acs{ms}
        & \textit{2020}
        \\
        John Bell 
        & \acl{ui}
        & Hierarchical inference and spoofing alarm in HVDC control systems
        & \acs{ms}
        & \textit{2020}
        \\
        Winfred Sowah 
        & \acl{ui}
        & Thermal behavior of cold plated storage cask for used light water reactor nuclear fuels
        & \acs{ms}
        & \textit{2019}
        \\
        D. Devin Imholte 
        & \acl{inl}
        & Conceptual design of the Advanced Test Reactor non-destructive examination system
        & \acs{ms}
        & \textit{2019}
        \\
        John Biersdorf 
        & \acl{ui}
        & Precipitation changes to Idaho National Laboratory over time
        & MEng
        & \textit{2018}
    \end{longtable}


    \renewcommand*{\arraystretch}{1.25}
    \begin{longtable}{m{1.25 in}L{1.25 in}L{2.50 in}p{0.30in}p{0.05in}}
        \multicolumn{4}{l}{\textbf{\acl{isu}}}\\
        \multicolumn{4}{l}{\textbf{Committee Member}}\\
        \multicolumn{4}{l}{\textit{Graduates}}\\
        Pedro Mena
        & \acl{isu}
        & Reactor transient classification using machine learning
        & \acs{phd}
        & \textit{2012}
        \\
        Pedro Mena
        & \acl{isu}
        & Auto Machine Learning Applications for Nuclear Reactors: Transient Identification, Model Redundancy and Security  
        & \acs{ms}
        & \textit{2019}
    \end{longtable}


\section*{Materials Developed}
    \begin{enumerate}[topsep=0pt,itemsep=0ex,partopsep=0ex,parsep=0ex,leftmargin=*,label=(\arabic*)]
        \item[]\textbf{Open Educational Resources} 
        \item[]Open source educational materials were compiled in order to produce an online textbook as a supplement to existing commercial textbooks. An online textbook allows flexibility to augment course content without requiring multiple textbooks. It also allows the educational content of a course to be more closely aligned with the desired learning outcomes. Two online texts have been developed -- \href{https://uidaho.pressbooks.pub/nuclearengineering}{Principles of Nuclear Engineering} and \href{https://uidaho.pressbooks.pub/riskassessment}{Risk Assessment}.
    \end{enumerate}
    \begin{enumerate}[topsep=7pt,itemsep=-0ex,partopsep=0ex,parsep=0ex,leftmargin=*,label=(\arabic*)]
        \item[]\textbf{Courses}
        \item[]\textit{NE450: Principles of Nuclear Engineering}
        \item[]\textit{NE514: Nuclear Safety} -- (adapted, with about 40\% new material)
        \item[]\textit{NE516: Nuclear Rules \& Regulations} -- (adapted, with about 40\% new material)
        \item[]\textit{NE527: Nuclear Materials Storage, Disposal, Transport} -- (under development)
        \item[]\textit{NE529: Risk Assessment}
        \item[]\textit{NE535: Criticality Safety I}
        \item[]\textit{NE585: Nuclear Fuel Cycles}
        \item[]\textit{NE587: Nuclear Facility Decommissioning} -- (collaboration with David Haar)
        \item[]\textit{CORS234: The Science of Engineering and Technology in the Modern World} -- (Nuclear physics module)
    \end{enumerate}


\section*{Scholarship Accomplishments}
    \begin{enumerate}[topsep=0pt,itemsep=-0ex,partopsep=0ex,parsep=0ex,leftmargin=*,label=(\arabic*)]
        \item[]\textbf{Publications, Exhibitions, Performances, Recitals}\hfill{\textit{\textsuperscript{+}Students}}
        \item[]\textbf{Refereed/Peer Reviewed Journal Publications}
    \end{enumerate}
    \begin{enumerate}[topsep=0pt,itemsep=0.75ex,partopsep=0ex,parsep=0ex,leftmargin=0.50in,rightmargin=0.50in,label=(\arabic*)]
        \item\bibentry{koe26a}.
        \item\bibentry{wai26a}.
        \item\bibentry{har26a}.
        \item\bibentry{ber25a}.
%        \item\bibentry{bau25a}.
        \item\bibentry{bau25b}.
        \item\bibentry{kof25a}.
        \item\bibentry{man24a}.
        \item\bibentry{cal23a}.
        \item\bibentry{men23a}.
        \item\bibentry{roo23a}.
        \item\bibentry{roo23b}.
        \item\bibentry{wai23a}.
        \item\bibentry{wid23a}.
        \item\bibentry{chr22a}.
        \item\bibentry{man22a}.
        \item\bibentry{men22a}.
        \item\bibentry{men22b}.
        \item\bibentry{red22a}.
        \item\bibentry{red22b}.
        \item\bibentry{wid22a}.
        \item\bibentry{dus21a}.
        \item\bibentry{dus21b}.
        \item\bibentry{men21a}.
        \item\bibentry{tac21a}.
        \item\bibentry{wid21a}.
        \item\bibentry{wid21b}.
        \item\bibentry{bor20a}.
        \item\bibentry{car20a}.
        \item\bibentry{car20b}.
        \item\bibentry{chr20a}.
        \item\bibentry{chr20b}.
        \item\bibentry{nov20a}.
        \item\bibentry{lee19a}.
        \item\bibentry{lee19b}.
        \item\bibentry{pet19a}.
        \item\bibentry{red18a}.
        \item\bibentry{bor17a}.
        \item\bibentry{lee17a}.
        \item\bibentry{bor16a}.
        \item\bibentry{bor14a}.
        \item\bibentry{bor14b}.
        \item\bibentry{bor14c}.
        \item\bibentry{bor13a}.
        \item\bibentry{bor13b}.
        \item\bibentry{bor11a}.
        \item\bibentry{bor08a}.
        \item\bibentry{bor08b}.
    \end{enumerate}
    \begin{enumerate}[topsep=7pt,itemsep=-0ex,partopsep=0ex,parsep=0ex,leftmargin=*,label=(\arabic*)]
        \item[]\textbf{Refereed/Peer Reviewed Conference Proceedings}
    \end{enumerate}
    \begin{enumerate}[topsep=0pt,itemsep=0.75ex,partopsep=0ex,parsep=0ex,leftmargin=0.50in,rightmargin=0.50in,label=(\arabic*)]
        \item\bibentry{cfbor26a}.
        \item\bibentry{cfwai26a}.
        \item\bibentry{cfkoe25a}.
        \item\bibentry{cfbal25a}.
        \item\bibentry{cfbor24a}.
        \item\bibentry{cfwai23a}.
        \item\bibentry{cfwhe23a}.
        \item\bibentry{cfroo22a}.
        \item\bibentry{cfhan21a}.
        \item\bibentry{cfric20a}.
        \item\bibentry{cfbor19a}.
        \item\bibentry{cfchr19a}.
        \item\bibentry{cfkol19a}.
        \item\bibentry{cfmac19a}.
        \item\bibentry{cfpet19a}.
        \item\bibentry{cfbor18a}.
        \item\bibentry{cfbor18b}.
        \item\bibentry{cfdus18a}.
        \item\bibentry{cflee18a}.
        \item\bibentry{cfbor17a}.
        \item\bibentry{cflee17a}.
        \item\bibentry{cfpet17a}.
        \item\bibentry{cfred17a}.
        \item\bibentry{cfred17b}.
        \item\bibentry{cfbor16a}.
        \item\bibentry{cfbor13a}.
        \item\bibentry{cfbor13b}.
        \item\bibentry{cfbor12a}.
        \item\bibentry{cfbor11a}.
        \item\bibentry{cfbor10a}.
        \item\bibentry{cfbor10b}.
        \item\bibentry{cfbor09a}.
        \item\bibentry{cfbor09b}.
        \item\bibentry{cfbor08a}.
        \item\bibentry{cfbor07a}.
        \item\bibentry{cfbor07b}.
        \item\bibentry{cfbor06a}.
        \item\bibentry{cfbor96a}.
    \end{enumerate}
    \begin{enumerate}[topsep=7pt,itemsep=-0ex,partopsep=0ex,parsep=0ex,leftmargin=*,label=(\arabic*)]
        \item[]\textbf{Book contributions}
    \end{enumerate}
    \begin{enumerate}[topsep=0pt,itemsep=0.75ex,partopsep=0ex,parsep=0ex,leftmargin=0.50in,rightmargin=0.50in,label=(\arabic*)]
        \item\bibentry{bmac19a}.
        \item\bibentry{bbor15a}.
    \end{enumerate}
    \begin{enumerate}[topsep=7pt,itemsep=-0ex,partopsep=0ex,parsep=0ex,leftmargin=*,label=(\arabic*)]
        \item[]\textbf{Non-Refereed Technical Reports}
    \end{enumerate}
    \begin{enumerate}[topsep=0pt,itemsep=0.75ex,partopsep=0ex,parsep=0ex,leftmargin=0.50in,rightmargin=0.50in,label=(\arabic*)]
        \item\bibentry{rwai23a}.
        \item\bibentry{rara21a}.
        \item\bibentry{rdus20a}.
        \item\bibentry{rbor19a}.
        \item\bibentry{rbor18a}.
        \item\bibentry{rahn15a}.
        \item\bibentry{rahn12a}.
        \item\bibentry{rhec12a}.
        \item\bibentry{rahn11a}.
        \item\bibentry{rahn10a}.
        \item\bibentry{rahn10b}.
        \item\bibentry{rbor10a}.
        \item\bibentry{rahn09a}.
        \item\bibentry{rbor07a}.
        \item\bibentry{rbor07b}.
        \item\bibentry{rbor07c}.
        \item\bibentry{rahn06a}.
    \end{enumerate}
    \begin{enumerate}[topsep=7pt,itemsep=-0ex,partopsep=0ex,parsep=0ex,leftmargin=*,label=(\arabic*)]
        \item[]\textbf{Conference Presentations}
    \end{enumerate}
    \begin{enumerate}[topsep=0pt,itemsep=0.75ex,partopsep=0ex,parsep=0ex,leftmargin=0.50in,rightmargin=0.50in,label=(\arabic*)]
        \item\bibentry{pbor26a}.
        \item\bibentry{pwai26a}.
        \item\bibentry{pkoe25a}.
        \item\bibentry{pbal25a}.
        \item\bibentry{pbor24a}.
        \item\bibentry{pwai23a}.
        \item\bibentry{pwhe23a}.
        \item\bibentry{proo22a}.
        \item\bibentry{proo22b}.
        \item\bibentry{phan21a}.
        \item\bibentry{pcut20a}.
        \item\bibentry{pric20a}.
        \item\bibentry{pbor19a}.
        \item\bibentry{pchr19a}.
        \item\bibentry{pkol19a}.
        \item\bibentry{pmac19a}.
        \item\bibentry{ppet19a}.
        \item\bibentry{ppet19b}.
        \item\bibentry{pbor18a}.
        \item\bibentry{pbor18b}.
        \item\bibentry{pdus18a}.
        \item\bibentry{phuf18a}.
        \item\bibentry{plee18a}.
        \item\bibentry{pbor17a}.
        \item\bibentry{plee17a}.
        \item\bibentry{ppet17a}.
        \item\bibentry{pred17a}.
        \item\bibentry{pred17b}.
        \item\bibentry{pbor16a}.
        \item\bibentry{pbor13a}.
        \item\bibentry{pbor13b}.
        \item\bibentry{pbor12a}.
        \item\bibentry{pbor11a}.
        \item\bibentry{pbor10a}.
        \item\bibentry{pbor10b}.
        \item\bibentry{pbor09a}.
        \item\bibentry{pbor09b}.
        \item\bibentry{pbor08a}.
        \item\bibentry{pbor07a}.
        \item\bibentry{pbor07b}.
        \item\bibentry{pbor06a}.
    \end{enumerate}
    \begin{enumerate}[topsep=15pt,itemsep=-0ex,partopsep=0ex,parsep=0ex,leftmargin=*,label=(\arabic*)]
        \item[]\textbf{Invited talks}
        \item[]\textit{\acl{iae}}
        \item[]Knowledge Management Assist Visits for Education and Training Providers\hfill{\textit{2021 virtual}}
    \end{enumerate}
    \begin{enumerate}[topsep=7pt,itemsep=-0ex,partopsep=0ex,parsep=0ex,leftmargin=*,label=(\arabic*)]
        \item[]\textit{\acl{ui} $\sq$ \acl{iff}}
        \item[]Strategies and success for ethical research\hfill{\textit{2018--19; 2022}}
        \item[]Vandal Advantage Graduate Student Orientation\hfill{\textit{2017--19}}
        \item[]P3/TRIO Upward Bound STEM Day Presentation\hfill{\textit{2015}}
    \end{enumerate}
    \begin{enumerate}[topsep=7pt,itemsep=-0ex,partopsep=0ex,parsep=0ex,leftmargin=*,label=(\arabic*)]
        \item[]\textit{Stanford University}
        \item[]Pre-Collegiate Summer Institutes\hfill{\textit{2014}}
    \end{enumerate}
    \begin{enumerate}[topsep=7pt,itemsep=-0ex,partopsep=0ex,parsep=0ex,leftmargin=*,label=(\arabic*)]
        \item[]\textbf{Seminars}
        \item[]\acl{utk} Nuclear Engineering Department\hfill{\textit{2019}}
    \end{enumerate}
    \begin{enumerate}[topsep=7pt,itemsep=-0ex,partopsep=0ex,parsep=0ex,leftmargin=*,label=(\arabic*)]
        \item[]\textbf{Lightning Talks}
        \item[]Future of Nuclear Waste Education -- Massachutsetts Institute of Technology\hfill{\textit{2022}}
        \item[]\acl{inl} Nuclear Science and Technology Collaborative Planning Meeting\hfill{\textit{2018}}
        \item[]\acl{inl} National and Homeland Security Collaborative Planning Meeting\hfill{\textit{2018}}
        \item[]\acl{inl} Energy \& Environment Collaborative Planning Meeting\hfill{\textit{2017}}
    \end{enumerate}


\section*{Grants and Contracts Awarded}
    \begin{enumerate}[topsep=0pt,itemsep=0.75ex,partopsep=0ex,parsep=0ex,leftmargin=0.50in,rightmargin=0.50in,label=(\arabic*)]
        \item Sean McBride (\acs{pi}) - Idaho State University, Dakota Roberson (co-\acs{pi}), R. A. Borrelli (co-\acs{pi}), Constantinos Kolias (co-\acs{pi}) - University of Idaho. Industrial Cyber Security Research and Training Laboratory. National Institute of Standards and Technology. \textbf{\$2,875,000}. 2024.08.01 - 2026.07.31. [\textit{non-competitive}]
        \item Michael Haney (\acs{pi}), R. A. Borrelli (co-\acs{pi}), Dakota Roberson (co-\acs{pi}), Constantinos Kolias (co-\acs{pi}) - University of Idaho, Ben Lampe (co-\acs{pi}), Sean McBride (co-\acs{pi}) - Idaho State University. Secure Cyberspace and Resilient Industrial Systems Workforce Development. \acl{igm} $\sq$ \acl{hrc}. \textbf{\$700,000.} 2024.07.01 - 2025.06.30. [\textit{Borrelli \acs{pi} 2024.07.01}]
        \item Kathleen Ara{\'u}jo (\acs{pi}), Cassie Koerner (co-\acs{pi}) - Boise State University, Stephanie Malin (co-\acs{pi}) - Colorado State University, Daniel Cardenas (co-\acs{pi}) - National Tribal Energy Association, R. A. Borrelli (co-\acs{pi}) - University of Idaho, Weston Eaton (co-\acs{pi}), Temple Stoellinger (Senior Personnel), Steven Smutko (Senior Personnel), Rachael Budowle (Senior Personnel) - University of Wyoming, Majia Nadesan (co-\acs{pi}) - Arizona State University, Julia Haggerty (co-\acs{pi}), Lee Spangler (Senior Personnel) - Montana State University, Denia Djoki{\'c} (co-\acs{pi}) University of Michigan, Sarah Robey (co-\acs{pi}) - Idaho State University. Common ground: Legitimacy in consent-based siting for interim nuclear waste storage. \acl{doe} Consent-Based Siting for Interim Storage Program - Community Engagement Opportunities. \textbf{\$2,000,000.} 2023.08.01 - 2025.07.31.
        \item Andrew Kliskey (\acs{pi}) - Idaho \acs{eps} Director, Karla Eitel (co-\acs{pi}), Alistair Smith (co-\acs{pi}) - University of Idaho, Donna Lybecker (co-\acs{pi}) - Idaho State University, Kathleen Ara{\'u}jo (co-\acs{pi}) Boise State University. \href{https://www.idahoepscor.org/i-crews}{RII Track-1: Idaho Community-engaged Resilience for Energy-Water Systems (I-CREWS)}. National Science Foundation \acs{eps}. \textbf{\$24,000,000.} 2023.09.01 - 2028.08.31.\footnote{Technical Proposal Writing Team.}\textsuperscript{,}\footnote{University of Idaho Research Team.}  [\textit{Modeling Group Lead 2025.09.01}] [\textit{Treasure Valley Site Planning Team 2026.01.16}]
        \item Michael Haney (\acs{pi}), R. A. Borrelli (co-\acs{pi}), Dakota Roberson (co-\acs{pi}), Constantinos Kolias (co-\acs{pi}) - University of Idaho, Ben Lampe (co-\acs{pi}), Sean McBride (co-\acs{pi}) - Idaho State University. Secure Cyberspace and Resilient Industrial Systems Workforce Development. Idaho Global Entrepreneurial Mission (IGEM) --  Higher Education Research Council \textbf{\$700,000.} 2023.07.01 - 2024.06.30.
        \item Michael Haney (\acs{pi}), R. A. Borrelli (co-\acs{pi}), Dakota Roberson (co-\acs{pi}), Constantinos Kolias (co-\acs{pi}) - University of Idaho, Ben Lampe (co-\acs{pi}), Sean McBride (co-\acs{pi}) - Idaho State University. Secure Cyberspace and Resilient Industrial Systems Workforce Development. Idaho Global Entrepreneurial Mission (IGEM) --  Higher Education Research Council \textbf{\$693,000.} 2022.07.01 - 2023.06.30.
        \item R. A. Borrelli (\acs{pi}), Michael Haney (co-\acs{pi}) - University of Idaho. Cyber-informed design, education, and training for cyberthreat resiliency with real-time nuclear reactor simulation. University of Idaho. Operation: Resubmission Support. \textbf{\$34,122.} 2022.04.30 - 2022.09.30.
        \item Thomas A. Ulrich (\acs{pi}) - Idaho National Laboratory, R. A. Borrelli (co-\acs{pi}) - University of Idaho. User evaluation of the NuScale simulator at the Center for Advanced Energy Studies. CAES programmatic funding. \textbf{\$50,000.} 2022.03.01 - 2022.09.30.
        \item R. A. Borrelli (\acs{pi}) - University of Idaho, Dennis D. Keiser, Jr., (co-\acs{pi}) - Idaho National Laboratory. Graduate Research Assistantship: Connecting U-Mo Fuel Processing, Microstructure, and Irradiation Performance. \textbf{\$23,823.} 2021.06.01-2022.01.31.
        \item R. A. Borrelli (\acs{pi}), Jason Barnes (Senior Adviser) - University of Idaho. Experimental determination of interactions between the radiation fields of Dragonfly’s MMRTG and Titan’s environment. Idaho NASA \acs{eps} Research Initiation Grant. \textbf{\$82,962}. 2021.05.01 - 2022.04.30.
        \item R. A. Borrelli (\acs{pi}) - University of Idaho. Online educational resources for nuclear engineering courses. University of Idaho Think Open Fellowship. \textbf{\$1200}. 2021.01.01 - 2021.05.31. 
    %    \item Athi Varuttamaseni (\acs{pi}), Shinjae Yoo (co-\acs{pi}) - Brookhaven National Laboratory, R. A. Borrelli (co-\acs{pi}) - University of Idaho. Adaptive control and monitoring platform for autonomous operation of advanced reactors. NEUP 20-19280. \textbf{\$1,000,000.} 2020.10.01 - 2023.09.30.
        \item Lee Ostrom (\acs{pi}), Richard N. Christensen, R. A. Borrelli, Haiyan Zhao (co-\acs{pi}s) - University of Idaho. ORED Fall 2019 EIS: Portable XFR for use in supporting material research. ORED Equipment and Infrastructure Support. \textbf{\$40,000.} 2019.12.01 - 2020.11.30
        \item R. A. Borrelli (\acs{pi}) - University of Idaho, Mark. D. DeHart (co-\acs{pi}) - Idaho National Laboratory. Application and enhancement of MAMMOTH depletion capabilities. \textbf{\$33,521.} 2020.01.13 - 2020.12.31.
        \item Richard N. Christensen (\acs{pi}), R. A. Borrelli, Michael G. McKellar, Michael Haney, David Arcilesi (co-\acs{pi}s) - University of Idaho, Richard Jacobson (co-\acs{pi}) Idaho State University. NuScale Simulator at the Center for Advanced Energy Studies. \acl{doe} Scientific Infrastructure Support for Consolidated Innovative Nuclear Research. \textbf{\$321,525.} 2019.10.01 - 2022.09.30. [\textit{\acs{pi} - NuScale Simulator Laboratory - 2022.01.07}]
        \item R. A. Borrelli (\acs{pi}) - University of Idaho, Dennis D. Keiser, Jr., (co-\acs{pi}) - Idaho National Laboratory. Graduate Research Assistantship: Connecting U-Mo Fuel Processing, Microstructure, and Irradiation Performance. \textbf{\$127,866.} 2018.10.01-2021.05.31.
        \item R. A. Borrelli (\acs{pi}), Richard N. Christensen (co-\acs{pi}) - University of Idaho, Brian J. Jaques (co-\acs{pi}) - Boise State University, Piyush Sabharwall (co-\acs{pi}) - Idaho National Laboratory, Mark Delligatti (co-\acs{pi}) - Table Rock, LLC, Sakae Casting USA, LLC (co-\acs{pi}). Modeling and design of borated aluminium cask for used fuel cooling. Idaho Global Entrepreneurial Mission (IGEM) - Idaho Commerce. \textbf{\$237,898.} 2018.01.01-2019.05.31.
        \item R. A. Borrelli (\acs{pi}) - University of Idaho, Dennis D. Keiser, Jr., (co-\acs{pi}) - Idaho National Laboratory. Graduate Research Assistantship: Connecting U-Mo Fuel Processing, Microstructure, and Irradiation Performance. \textbf{\$36,180.} 2017.11.01-2018.05.31.
        \item R. A. Borrelli (\acs{pi}), Lee Ostrom (Senior Advisor) - University of Idaho, Stephen G. Johnson (Senior Advisor) - Idaho National Laboratory. Performance assessment of americium as fuel in radioisotope thermoelectric generators for deep space exploration. Idaho NASA \acs{eps} Research Initiation Grant. \textbf{\$55,000.} 2017.08.01-2018.04.30.
        \item Kelley M. Verner (\acs{pi}), R. A. Borrelli, Marc T. Skinner, Emma K. Redfoot, Jieun Lee, Seth Dustin, John Peterson (co-\acs{pi}s) - University of Idaho. Increasing the Go-on Rate in Southeast Idaho Through the Nexus of Food, Energy, and Water. University of Idaho Vandals Big Ideas Project. \textbf{\$23,984.} 2017.07.01-2018.06.30.
        \item R. A. Borrelli (\acs{pi}) - University of Idaho, Jason Hales (co-\acs{pi}) - Idaho National Laboratory. Graduate Research Assistantship: Idaho National Laboratory Code Documentation. \textbf{\$35,435.} 2016.10.01-2017.06.30.
        \item Vivek Utgikar (\acs{pi}), Fatih Aydogan, Krishnan Raja, Raghunath Kanakala, R. A. Borrelli, Haiyan Zhao, Matthew Swenson (co-\acs{pi}s) - University of Idaho. University of Idaho Nuclear Engineering Faculty Development Program. United States Regulatory Commission Faculty Development Grant. \textbf{\$434,048.} 2015.09.29 - 2019.09.30.
    \end{enumerate}
    \begin{enumerate}[topsep=7pt,itemsep=0ex,partopsep=0ex,parsep=0ex,leftmargin=*,label=(\arabic*)]
        \item[]\textbf{Major Equipment Acquisitions}
        \item[]\acl{wsc} \acl{pwr} \acl{npp} Simulator.
    \end{enumerate}

\section*{Honors and Awards}
    \begin{enumerate}[topsep=0pt,itemsep=-0ex,partopsep=0ex,parsep=0ex,leftmargin=*,label=(\arabic*)]
        \item[]\textbf{\acl{ui} \acl{ans} Student Section}
        \item[]Samuel Glasstone Award for Public Service -- Second Place\hfill{\textit{2019}}
        \item[]Certificate of Distinction\hfill{\textit{2019--}}
    \end{enumerate}
    \begin{enumerate}[topsep=7pt,itemsep=-0ex,partopsep=0ex,parsep=0ex,leftmargin=*,label=(\arabic*)]
        \item[]\textbf{University of California --Berkeley}
        \item[]Nuclear Engineering Department Block Grant Fellowship\hfill{\textit{2005}}
        \item[]Outstanding Graduate Student Instructor Award\hfill{\textit{2003}}
        \item[]Katherina S. DeSharton Fellowship\hfill{\textit{1999}}
        \item[]Hamilton Family Memorial Fellowship\hfill{\textit{1999}}
    \end{enumerate}
    \begin{enumerate}[topsep=7pt,itemsep=-0ex,partopsep=0ex,parsep=0ex,leftmargin=*,label=(\arabic*)]
        \item[]\textbf{\acl{doe} $\sq$ \acl{ocm}}
        \item[]Civilian Radioactive Waste Management Fellowship\hfill{\textit{2000--04}}
    \end{enumerate}
    \begin{enumerate}[topsep=7pt,itemsep=-0ex,partopsep=0ex,parsep=0ex,leftmargin=*,label=(\arabic*)]
        \item[]\textbf{\acl{nei}}
        \item[]National Academy for Nuclear Training Fellowship\hfill{\textit{1999}}
    \end{enumerate}


\section*{Service}
    \begin{enumerate}[topsep=0pt,itemsep=-0ex,partopsep=0ex,parsep=0ex,leftmargin=*,label=(\arabic*)]
        \item[]\textbf{Major Committee Assignments}
        \item[]\textbf{\acl{ui}}
        \item[]\textit{Chair}
        \item[]Faculty Affairs Committee\hfill{\textit{2025--26}}
        \item[]Promotion \& Tenure -- \acl{nem} \hfill{\textit{2023}}
        \item[]Third Year Promotion \& Tenure -- \acl{nem} \hfill{\textit{2021}}
        \item[]\textit{Member}
        \item[]University Level Promotions and Tenure Silver Committee \hfill{\textit{2026}}
        \item[]University Level Promotions and Tenure Gold Committee \hfill{\textit{2025}}
        \item[]Center for Excellence in Teaching and Learning Accessibility Advisory Committee \hfill{\textit{2025--6}}
        \item[]Faculty Compensation Committee \hfill{\textit{2025--6}}
        \item[]Promotion -- Chemical \& Biological Engineering \hfill{2025}
        \item[]Strategic Planning Focus Group \hfill{\textit{2025}}
        \item[]Faculty Senate -- College of Engineering \hfill{\textit{2024--27}}
        \item[]Scientific Misconduct Committee \hfill{\textit{2025--26}}
        \item[]Faculty Affairs Committee\hfill{\textit{2024--26}}
        \item[]Faculty Appeals Hearing Board Committee\hfill{\textit{2024--25}}
        \item[]Promotion \& Tenure, College of Engineering \hfill{\textit{2020--23}}
        \item[]\acl{nrc} Student Fellowship Oversight Committee
        \item[]\acl{nem} Bylaws Committee
        \item[]Graduate Faculty
        \item[]\acl{nem} Admissions Committee
        \item[]Commencement Speaker Committee -- \acl{iff}
        \item[]\textit{Organizer}
        \item[]American Nuclear Society Graduate School Fair \hfill{\textit{2020 -- 22}}
        \item[]\acs{nna} Fellowship Information Symposia for Idaho Universities \hfill{\textit{2017 -- }}
    \end{enumerate}
    \begin{enumerate}[topsep=7pt,itemsep=-0ex,partopsep=0ex,parsep=0ex,leftmargin=*,label=(\arabic*)]
        \item[]\textbf{Faculty Search Committees}
        \item[]\textit{Chair}
        \item[]\acl{nem}\hfill{\textit{2024}}
        \item[]\textit{Member}
        \item[]Grants \& Contracts Specialist II\hfill{\textit{2024}}
        \item[]\acl{nem}\hfill{\textit{2023}}
        \item[]Nuclear Engineering\hfill{\textit{2022}}
        \item[]Mechanical/Nuclear Engineering\hfill{\textit{2017}}
    \end{enumerate}
    \begin{enumerate}[topsep=7pt,itemsep=-0ex,partopsep=0ex,parsep=0ex,leftmargin=*,label=(\arabic*)]
        \item[]\textbf{\acl{isu}}
        \item[]\textbf{Faculty Search Committees}
        \item[]Nuclear Engineering\hfill{\textit{2018}}
    \end{enumerate}
    \begin{enumerate}[topsep=15pt,itemsep=-0ex,partopsep=0ex,parsep=0ex,leftmargin=*,label=(\arabic*)]
        \item[]\textbf{Professional and Scholarly Organizations}
        \item[]\textbf{National}
        \item[]\textit{\acl{ans}}
        \item[]National Program Screening Subcommittee\hfill{\textit{2022--}}
        \item[]Local Sections Committee
        \item[]\hspace*{0.15in}Member-at-Large \hfill{\textit{2025--28}}
        \item[]Fuel Cycle \& Waste Management Division\hfill{\textit{2015--}}
        \item[]\hspace*{0.15in}Executive Committee \hfill{\textit{2018--21}}
        \item[]\hspace*{0.15in}Student Support Subcommittee \hfill{\textit{2025}}
        \item[]Nonproliferation Policy Division\hfill{\textit{2015--}}
        \item[]\hspace*{0.15in}Executive Committee \hfill{\textit{2019--22}}
        \item[]Student Sections Committee\hfill{\textit{2015--}}
        \item[]\hspace*{0.15in}Executive Committee \hfill{\textit{2018--21; 2024--27}}
    \end{enumerate}
    \begin{enumerate}[topsep=7pt,itemsep=-0ex,partopsep=0ex,parsep=0ex,leftmargin=*,label=(\arabic*)]
        \item[]\textit{Tau Beta Pi Engineering Society}\hfill{\textit{1996}}
    \end{enumerate}
    \begin{enumerate}[topsep=7pt,itemsep=-0ex,partopsep=0ex,parsep=0ex,leftmargin=*,label=(\arabic*)]
        \item[]\textbf{Regional}
        \item[]\textit{Idaho Section of the \acl{ans}}
        \item[]Treasurer\hfill{\textit{2022--}}
        \item[]Board of Directors\hfill{\textit{2018; 2020}}
        \item[]Nuclear Science Week Planning Committee\hfill{\textit{2017}}
        \item[]Community Service Committee\hfill{\textit{2015--}}
    \end{enumerate}
    \begin{enumerate}[topsep=7pt,itemsep=-0ex,partopsep=0ex,parsep=0ex,leftmargin=*,label=(\arabic*)]
        \item[]\textbf{\acl{ui}}
        \item[]Faculty Advisor -- \acl{ans} Student Section\hfill{\textit{2015--}}
    \end{enumerate}
    \begin{enumerate}[topsep=15pt,itemsep=-0ex,partopsep=0ex,parsep=0ex,leftmargin=*,label=(\arabic*)]
        \item[]\textbf{Peer Reviewer}
        \item[]\textbf{\acl{ans}}
        \item[]Advances in Nuclear Nonproliferation Technology and Policy
        \item[]Annual \& Winter Meetings 
        \item[]Fuel Cycle and Waste Management Division John Randall Scholarship 
        \item[]\hspace*{0.15in} Chair\hfill{\textit{2022 -- 23}} 
        \item[]Instrumentation, Control \& Human-Machine Interface Technologies
        \item[]International High-Level Radioactive Waste Management Conference
        \item[]Student Conferences
    \end{enumerate}
    \begin{enumerate}[topsep=7pt,itemsep=-0ex,partopsep=0ex,parsep=0ex,leftmargin=*,label=(\arabic*)]
        \item[]\textbf{Journals}
        \item[]Advances in Engineering Software 
        \item[]Annals of Nuclear Energy 
        \item[]Energy Science \& Engineering 
        \item[]IEEE Transactions on Nuclear Science 
        \item[]International Journal of Nuclear Energy 
        \item[]Nuclear Engineering and Design 
        \item[]Nuclear Engineering and Technology 
        \item[]Progress in Nuclear Energy 
    \end{enumerate}
    \begin{enumerate}[topsep=7pt,itemsep=-0ex,partopsep=0ex,parsep=0ex,leftmargin=*,label=(\arabic*)]
        \item[]\textbf{Government}
        \item[]Nuclear Science User Facilities 
        \item[]\acs{doe} Advanced Research Projects Agency-Energy Concept Papers
        \item[]\acs{doe} Advanced Research Projects Agency-Energy Review Panel
        \item[]\acs{doe} Office of Nuclear Energy, Consolidated Innovative Nuclear Research 
        \item[]\acs{doe} SBIR/STTR Phase I Release 2 
        \item[]\acs{doe} SBIR/STTR Phase II Release 1 
        \item[]\acs{doe} SBIR/STTR Phase II Release 2 
    \end{enumerate}
    \begin{enumerate}[topsep=7pt,itemsep=-0ex,partopsep=0ex,parsep=0ex,leftmargin=*,label=(\arabic*)]
        \item[]\textbf{Other}
        \item[]Louisiana Board of Regents Support Fund R \& D Grants Programs
        \item[]Clays in Natural \& Engineered Barriers for Radioactive Waste Confinement
        \item[]Bonneville Power Administration, Office of Technology Innovation 
        \item[]John Wiley \& Sons, Inc. 
        \item[]Khalifa University of Science and Technology 
\end{enumerate}
    \begin{enumerate}[topsep=15pt,itemsep=-0ex,partopsep=0ex,parsep=0ex,leftmargin=*,label=(\arabic*)]
        \item[]\textbf{Professional Activities}
        \item[]\textbf{Conference Committees}
        \item[]\textit{\acl{ans}}
        \item[]Technical Program Committee, International High Level Radioactive Waste Management, Washington, D.C.\hfill{\textit{2025}}
        \item[]Technical Program, Advances in Nuclear Nonproliferation Technology and Policy, Orlando, Florida\hfill{\textit{2016; 2018}}
        \item[]\hspace*{0.15in} Co-Chair\hfill{\textit{2018}}
    \end{enumerate}
    \begin{enumerate}[topsep=7pt,itemsep=-0ex,partopsep=0ex,parsep=0ex,leftmargin=*,label=(\arabic*)]
        \item[]\textbf{Technical Session Organizer}
        \item[]\textit{\acl{ans}}
        \item[]Cybersecurity for Nuclear Installations, Washington, D. C.\hfill{\textit{2019; 2021}}
        \item[]\hspace*{0.15in}co-Organizer with Prof. Jamie B. Coble, University of Tennessee--Knoxville
    \end{enumerate}
    \begin{enumerate}[topsep=7pt,itemsep=-0ex,partopsep=0ex,parsep=0ex,leftmargin=*,label=(\arabic*)]
        \item[]\textbf{Technical Session Chair}
        \item[]\textit{\acl{ans}}
        \item[]Program Development, Strategy, and Policy: II, International High Level Radioactive Waste Management, Washington, D. C.\hfill{\textit{2025}}
        \item[]Cybersecurity for Nuclear Installations, Anaheim, California\hfill{\textit{2022}}
        \item[]Technology and Policy Advancements in Nuclear Nonproliferation Anaheim, California\hfill{\textit{2022}}
        \item[]Cybersecurity for Nuclear Installations, Washington, D. C.\hfill{\textit{2019; 2021}}
        \item[]Spent Fuel Storage and Transportation, Washington, D. C.\hfill{\textit{2019}}
        \item[]Data Synthesis for Pyroprocessing Safeguards, Advances in Nuclear Nonproliferation Technology and Policy, Orlando Florida\hfill{\textit{2018}}
        \item[]Prof. Joonhong Ahn Memorial Session, International High-Level Radioactive Waste Management, Charlotte, North Carolina\hfill{\textit{2017}}
        \item[]Nonproliferation Policy, Concepts, and Approaches: Treaty verification regimes, State-level Concepts, and Fuel Cycle Analysis, Advances in Nuclear Nonproliferation Technology and Policy, Santa Fe, New Mexico\hfill{\textit{2016}}
        \item[]Security, Safeguards, and Non-proliferation, International High-Level Radioactive Waste Management, Albuquerque, New Mexico\hfill{\textit{2011}}
        \item[]Used Fuel Recycling Technologies--I, International Congress on Advances in Nuclear Power Plants, San Diego, California\hfill{\textit{2010}}
        \item[]Engineered Systems and Transport Processes, International High-Level Radioactive Waste Management, Las Vegas, Nevada\hfill{\textit{2008}}
        \item[]\textit{\acl{bsu} \acl{epi}}
        \item[]Diverse Approaches to Addressing Decarbonization, Energy Policy\hfill{\textit{2021 virtual}}
        \item[]\textit{University of Tokyko}
        \item[]Challenges in Nuclear Waste Disposal: Sociological Aspects and Technical Approaches, Global Center of Excellence\hfill{\textit{2007}}
    \end{enumerate}
    \begin{enumerate}[topsep=7pt,itemsep=-0ex,partopsep=0ex,parsep=0ex,leftmargin=*,label=(\arabic*)]
        \item[]\textbf{Panels}
        \item[]\textit{Moderator}
        \item[]\textit{\acl{ans}}
        \item[]Future of Cybersecurity in Nuclear Installations\hfill\textit{2021} 
        \item[]\textit{Panelist}
        \item[]\textit{\acl{ans}}
        \item[]Future of Cybersecurity in Nuclear Installations\hfill\textit{2021} 
        \item[]Taking Care of You: Student-Mentor Relationships\hfill{\textit{2021}}
        \item[]A great mentoring situation can pave the way for success out of undergraduate or graduate school. Likewise, a poor or unhealthy mentoring relationship can cause lasting negative consequences on the students. Many times these interactions go undiscussed, and one goal of this workshop is to empower students to talk about and take charge of unhealthy mentoring relationships. 
        \item[]\textit{\acl{ui}}
        \item[]Idaho Open Education Week Think Open Fellowship\hfill{\textit{2021}}
        \item[]Think Open Fellows an incubator program that awards 6 fellowships each year to support faculty and graduate students in transitioning a course from a standard text to open or extremely low cost course materials.
        \item[]\textit{\acl{bsu} \acl{epi} Power Talks}
        \item[]Economic Opportunities and Challenges for Idaho with Low Carbon Energy\hfill{\textit{2022}}
    \end{enumerate}
    \begin{enumerate}[topsep=7pt,itemsep=-0ex,partopsep=0ex,parsep=0ex,leftmargin=*,label=(\arabic*)]
        \item[]\textbf{Workshops}
        \item[]\textbf{\acl{icr}}
        \item[]\textit{Machine Learning, Computational Modeling, Risk Assessment}\hfill{\textit{2024}}
        \item[]\textit{Presenter}\hfill[\textit{virtual}]
        \item[] The workshop is designed to provide an introductory overview of machine learning, including what machine learning is, different types of learning, and machine learning in practice. The workshop will provide machine learning examples used in water and energy research, followed by discussion. Additionally, the workshop will feature discussions on energy modeling with a focus on timescales, and an exploration of risk, including its dynamic nature and examples from the literature. The objective of the workshops are to provide an opportunity to enhance participant understanding of machine learning, fostering collaboration and knowledge exchange within the \acs{icr} community. 
    \end{enumerate}
    \begin{enumerate}[topsep=7pt,itemsep=-0ex,partopsep=0ex,parsep=0ex,leftmargin=*,label=(\arabic*)]
        \item[]\textbf{Future of Nuclear Waste Education}
        \item[]\textit{Clemson University}\hfill{\textit{2023}}
        \item[]\textit{Presenter}
        \item[]Understanding and managing nuclear waste requires a broad knowledge from nuclear engineering to civil and environmental engineering. In parallel, broader system-level analyses as well as societal/regulatory contexts are necessary to inform energy policies from the nuclear waste and environmental perspectives. Energy system or nuclear fuel cycle analysis quantifies waste generation per unit energy output, and their economical and environmental impacts throughout the life cycles of nuclear energy. Historical and regulatory contexts are also important to understanding nuclear waste in the contexts of general hazardous waste management. A challenge is posed by the extreme interdisciplinary nature of the subject of nuclear waste management— which makes it difficult to cover all the important topics in one, single institution. Such a knowledge gap is a major bottleneck when transformational changes are needed to prepare for new types of wastes from advanced reactors as well as to develop new waste processing technologies such as pyroprocessing. Although each individual institution can have its strengths and characteristics, it is critical to create community efforts such that the next generation workforce – interacting with the public and policy makers – has current and comprehensive knowledge about nuclear waste and its environmental impacts. This workshop will develop the blueprint of educating next generation engineers/scientists in nuclear waste as well as broader nuclear and environmental engineers. Although one student may not need to have all the knowledge, there is a minimum set of knowledge that one should have as a future scientist or engineer involved in nuclear waste and nuclear energy.
    \end{enumerate}
    \begin{enumerate}[topsep=7pt,itemsep=-0ex,partopsep=0ex,parsep=0ex,leftmargin=*,label=(\arabic*)]
        \item[]\textbf{Nuclear Advocacy and Communications Training}
        \item[]\textbf{Generation Atomic}
        \item[]\textit{Idaho Falls}\hfill{\textit{2018}}
        \item[]\textit{Facilitator}
        \item[]Opposition groups claim nuclear power plants are unsafe. Recently, the U.S. nuclear power industry has been characterized as too expensive and dangerous when compared to other energy sources. As members of the nuclear community, we know that the success of nuclear energy has never been more important to ensuring a positive future for the world – but what can we do to make a difference? This workshop will leave participants energized to tell today’s nuclear power story and be well-equipped with the tools to do so. Convincing others about the benefits of nuclear involves more than just laying out the facts. Thoughtful and personal storytelling bridges gaps when speaking those who are unfamiliar with the technology by explaining the personal and moral reasons that we work in this field. Telling our personal stories and motivations for working in nuclear creates common ground from which we can better explain nuclear’s benefits: whether it’s as a mother talking to a father, a surfer talking to a skier, or a cook talking to a conservationist, the human stories that nuclear makes possible are our strongest tools when speaking to the public. The most effective nuclear advocacy takes place at the interpersonal level when we strike up conversations with peers and even better, strangers. Because you can never know who it might be sitting across from you at that dinner party or next to you on the airplane, it’s important to practice having open, considerate conversations with people of all backgrounds. 
    \end{enumerate}
    \begin{enumerate}[topsep=7pt,itemsep=-0ex,partopsep=0ex,parsep=0ex,leftmargin=*,label=(\arabic*)]
        \item[]\textbf{Modeling, Experimentation, Validation Summer School}
        \item[]\textit{\acl{inl}}\hfill{\textit{2017}}
        \item[]\textit{Mentor}
        \item[]The MeV Summer School provides enhanced training for engineers and applied scientists involved in design, licensing, and operation of current and advanced nuclear reactor systems through a multi-faceted learning approach of lectures, tours, and hands-on activities. The school is being organized through the cooperation of national laboratories, industry, government agencies, and universities that share the goal of building a strong workforce to support global nuclear expansion. The faculty will be drawn from the top experts in academia, industry, and government. The general organization and conduct of the school will be overseen by an international board of senior experts. A local secretariat will provide technical, logistical and administrative support to students and faculty. It is the aim of the school to foster the development of a next-generation network of scientists and engineers capable of advancing nuclear energy in the 21st century through integrated modeling and experimentation.
    \end{enumerate}
    \begin{enumerate}[topsep=7pt,itemsep=0ex,partopsep=0ex,parsep=0ex,leftmargin=*,label=(\arabic*)]
        \item[]\textbf{Symposia}
        \item[]\textit{Organizer}
        \item[]\textit{\acl{cae}}
        \item[]Nuclear cybersecurity research initiatives\hfill{\textit{2017}}
        \item[]Nuclear cybersecurity research focus areas\hfill{\textit{2016}}
        \item[]\textit{Stanford University}
        \item[]\textit{Center for International Security and Cooperation}
        \item[]Technical implications of nuclear energy system options\hfill{\textit{2011}}
    \end{enumerate}


\section*{Outreach}
    \begin{enumerate}[topsep=0pt,itemsep=0ex,partopsep=0ex,parsep=0ex,leftmargin=*,label=(\arabic*)]
        \item[]\textbf{\acl{ui}}
        \item[]\acl{ui}, \acl{usu}, \acl{bsu} \acl{ans} Student Social \hfill{\textit{2022}}
        \item[]\acl{ans} Diversity and Inclusion Committee Sponsorship \hfill{\textit{2020--}}
        \item[]\acl{usu} Graduate School Fair \hfill{\textit{2019}}
        \item[]Montana Tech Career Fair \hfill{\textit{2018--22}}
        \item[]\acl{ui} Moscow Campus Recruiting \hfill{\textit{2017--}}
        \item[]Live After 5 Idaho Falls \hfill{\textit{2017}}
        \item[]\acl{doe} consent based siting meeting -- Boise\hfill{\textit{2016}}
        \item[]PHYSOR \acl{ui} Idaho Sponsorship \hfill{\textit{2016}}
        \item[]\acl{bsu} Nuclear Research Summit \hfill{\textit{2016}}
        \item[]\acl{ui} \& \acl{byu} American Nuclear Society Student Social \hfill{\textit{2016; 2018}}
        \item[]\acl{byu}--Idaho Career Fair \hfill{\textit{2015}}
    \end{enumerate}
    \begin{enumerate}[topsep=7pt,itemsep=-0ex,partopsep=0ex,parsep=0ex,leftmargin=*,label=(\arabic*)]
        \item[]\textbf{Regional}
        \item[]\textbf{Idaho Section of the \acl{ans}}
        \item[]\textit{Highway Cleanup} \hfill{\textit{2016--}}
        \item[]Garbage cleanup of Miles 122-124 on Interstate 15 biannually
        \item[]\textit{Smoke Detector Donation Program}
        \item[]Since 2008, we have worked with more than 75 fire departments to donate more than 6200 smoke detectors to Idaho residents. From 2016, we donated over 2300 smoke detectors to Arimo, Ashton, Bancroft, Bear Lake, Clear Creek, Declo, Downey, Driggs, Grace, Hamer, Lava Hot Springs, Roberts, Shelley, and Soda Springs, across southeastern Idaho; Centerville, Clear Creek, Gem County, Horseshoe Bend, Idaho City, Lowman, Placerville and Valley of the Pines in Boise County; and north to the panhandle in Coeur D'Alene, Mullan, Shoshone, North Side, and West Pend, as well as nonprofits such as Habitat for Humanity and Club Inc. In 2022, we were pleased to be able to expand into Western Wyoming to Teton County.
    \end{enumerate}
    \begin{enumerate}[topsep=7pt,itemsep=-0ex,partopsep=0ex,parsep=0ex,leftmargin=*,label=(\arabic*)]
        \item[]\textit{College of Eastern Idaho Machine Tool Technology Advisory Board}\hfill{\textit{2018}}
    \end{enumerate}


\section*{Professional Development}
    \begin{enumerate}[topsep=0pt,itemsep=-0ex,partopsep=0ex,parsep=0ex,leftmargin=*,label=(\arabic*)]
        \item[]\textbf{(Re)Connecting with the 6Rs}
        \item[]\textbf{A Workshop Series on Ethical Native-Engaged Research Respect, Relationality, Responsibility, Representation, Relevance, and Reciprocity}
        \item[]\hfill{\textit{2026}}
        \item[]This Reciprocal Research Workshop series, funded by the \acs{nsf} \acs{eps} \acs{icr} grant, employs the 6 Rs (Respect, Relationality, Responsibility, Representation, Relevance, Reciprocity) framework for exploring the dynamics, logistics, and practical considerations of conducting or engaging in collaborative research with, in, and by Native Nations. 
    \end{enumerate}
    \begin{enumerate}[topsep=0pt,itemsep=-0ex,partopsep=0ex,parsep=0ex,leftmargin=*,label=(\arabic*)]
        \item[]\textbf{Future of Nuclear Waste Education Workshop}
        \item[]\textit{Vanderbilt University}\hfill{\textit{2024}}
        \item[]\textit{Clemson University}\hfill{\textit{2023}}
        \item[]\textit{Massachusetts Institute of Technology}\hfill{\textit{2024}}
        \item[]Understanding and managing nuclear waste requires a broad knowledge from nuclear engineering to civil and environmental engineering. In parallel, broader system-level analyses as well as societal/regulatory contexts are necessary to inform energy policies from the nuclear waste and environmental perspectives. Energy system or nuclear fuel cycle analysis quantifies waste generation per unit energy output, and their economical and environmental impacts throughout the life cycles of nuclear energy. Historical and regulatory contexts are also important to understanding nuclear waste in the contexts of general hazardous waste management. A challenge is posed by the extreme interdisciplinary nature of the subject of nuclear waste management— which makes it difficult to cover all the important topics in one, single institution. Such a knowledge gap is a major bottleneck when transformational changes are needed to prepare for new types of wastes from advanced reactors as well as to develop new waste processing technologies such as pyroprocessing. Although each individual institution can have its strengths and characteristics, it is critical to create community efforts such that the next generation workforce – interacting with the public and policy makers – has current and comprehensive knowledge about nuclear waste and its environmental impacts. This workshop will develop the blueprint of educating next generation engineers/scientists in nuclear waste as well as broader nuclear and environmental engineers. Although one student may not need to have all the knowledge, there is a minimum set of knowledge that one should have as a future scientist or engineer involved in nuclear waste and nuclear energy.
    \end{enumerate}
    \begin{enumerate}[topsep=0pt,itemsep=-0ex,partopsep=0ex,parsep=0ex,leftmargin=*,label=(\arabic*)]
        \item[]\textbf{International Workshop on Siting of Radioactive Waste Facilities} 
        \item[]\textbf{\acl{nwb}}
        \item[]\textit{Idaho Falls}\hfill{\textit{2023}}
        \item[]The mission of the U.S. \acl{nwb} is to perform unbiased and ongoing technical and scientific peer review of \acl{doe} nuclear waste management activities. The \acs{nwb} makes an essential contribution to increasing confidence in the scientific process and to informing, from a technical and scientific perspective, decisions on nuclear waste management. The \acs{nwb} provides objective technical information to Congress, the Administration, \acs{doe}, government and non-government organizations, and the public on a wide-range of issues related to spent nuclear fuel and high-level waste management and disposition.
    \end{enumerate}
    \begin{enumerate}[topsep=7pt,itemsep=-0ex,partopsep=0ex,parsep=0ex,leftmargin=*,label=(\arabic*)]
        \item[]\textbf{\acl{nsf} Grant Development Conference}
        \item[]\textit{Los Angeles}\hfill{\textit{2019}}
        \item[]\textit{Portland}\hfill{\textit{2016}}
        \item[]Key officials representing program directorates, administrative offices, Office of General Counsel, and Office of the Inspector General will participate in this two-day conference. The conference is considered a must, particularly for new faculty, researchers, educators and administrators who want to gain insight into a wide range of important and timely issues including: the state of current funding; the proposal and award process; and current and recently updated policies.
    \end{enumerate}
    \begin{enumerate}[topsep=7pt,itemsep=-0ex,partopsep=0ex,parsep=0ex,leftmargin=*,label=(\arabic*)]
        \item[]\textbf{Collaborative Open Source Curriculum Development Workshop}
        \item[]\textit{University of Illinois Urbana -- Champaign}\hfill{\textit{2017; 2018}}
        \item[]This workshop concluded a year of faculty interaction at six universities to develop curricula for courses common across the same disciplines at multiple universities in order to reduce the amount of time that any individual professor spends on creating what is essentially duplicate materials. The method proposed in this workshop is based on open source software development, where code is shared in online repositories, reviewed by peers, and contributed to the main project.
    \end{enumerate}
    \begin{enumerate}[topsep=7pt,itemsep=-0ex,partopsep=0ex,parsep=0ex,leftmargin=*,label=(\arabic*)]
        \item[]\textbf{PyNE Summit}
        \item[]\textit{University of Illinois Urbana -- Champaign}\hfill{\textit{2017}}
        \item[]PyNE is a suite of tools to aid in computational nuclear science and engineering. PyNE seeks to provide native implementations of common nuclear algorithms, as well as Python bindings and I/O support for other industry standard nuclear codes.
    \end{enumerate}
    \begin{enumerate}[topsep=7pt,itemsep=-0ex,partopsep=0ex,parsep=0ex,leftmargin=*,label=(\arabic*)]
        \item[]\textbf{Cyber Security Course for Safeguards Practitioners}
        \item[]\textit{\acl{pnl}}\hfill{\textit{2018}}
        \item[]The course is designed for early to mid-career safeguards practitioners (technical instrument developers, instrument users, policy advisors, etc.) who will benefit from a greater understanding of cyber security threats and how to reduce risks to safeguards systems and processes. This 3.5 day course is designed to teach safeguards and cybersecurity experts how to recognize and mitigate potential cybersecurity vulnerabilities in emerging safeguards instrumentation, information systems, and conduct of operations. This training features classroom style learning opportunities through hands-on exercises and provides plenty of time for questions and discussion. Participants will learn new cyber security skills, use cyber security tools, and collaborate with one another and cyber experts to resolve challenges. They will gain an understanding of cyber security principles and a better awareness of cyber risks associated with safeguards systems. Exercises will include puzzles, exploits of attended and unattended monitoring systems, blended physical and cyber attacks on fictional nuclear facilities, and network defense.
    \end{enumerate}
    \begin{enumerate}[topsep=7pt,itemsep=-0ex,partopsep=0ex,parsep=0ex,leftmargin=*,label=(\arabic*)]
        \item[]\textbf{Safeguards Laboratory Day}
        \item[]\textit{\acl{pnl}}\hfill{\textit{2017--19}}
        \item[]Students and faculty from the \acl{ui} invited to \acl{pnl} to learn about the research activities at the laboratory. The day also included hands-on safeguards and security experiments conducted at laboratory facilities, such as materials accounting and vehicle searches.
    \end{enumerate}
    \begin{enumerate}[topsep=7pt,itemsep=-0ex,partopsep=0ex,parsep=0ex,leftmargin=*,label=(\arabic*)]
        \item[]\textbf{Next Generation Safeguards Initiative Summer Course}
        \item[]\textit{\acl{pnl}}\hfill{\textit{2016}}
        \item[]This course, offered through the \acs{doe}/\acs{nna} Next Generation Safeguards Initiative, covers major international safeguards procedures currently in use in \acs{iae} member nations. Daily lectures were supplemented with hands-on safeguards activities conducted by \acs{iae} safeguards inspectors and researchers. Participants included faculty, postdoctorate researchers, and graduate students.
    \end{enumerate}
    \begin{enumerate}[topsep=7pt,itemsep=-0ex,partopsep=0ex,parsep=0ex,leftmargin=*,label=(\arabic*)]
        \item[]\textbf{Collaborative Research Planning Meeting}
        \item[]\textit{\acl{inl}}\hfill{\textit{2017; 2018}}
        \item[]The \acl{cae}is a catalyst for collaborative projects focused on energy research, achieved via connections between the \acs{cae} entities -– \acl{inl}, \acl{bsu}, \acl{isu}, and the \acl{ui}. Through a series of strategic planning meetings, \acs{cae} leadership aims to develop a set of focused research directions. The meeting is intended to establish new collaboration, along with a list of prioritized goals and actions items that will steer internal research investments with the intent of growing sustainable, externally-funded programs. The planning meeting will focus on a strategic area tied to one or more \acl{inl} directorates and will bring together the appropriate \acl{inl} and university leadership and researchers.
    \end{enumerate}
    \begin{enumerate}[topsep=7pt,itemsep=-0ex,partopsep=0ex,parsep=0ex,leftmargin=*,label=(\arabic*)]
        \item[]\textbf{Intermountain Energy Summit}
        \item[]\textit{Idaho Falls Post Register}\hfill{\textit{2015--17}}
        \item[]This summit is held annually and covers energy issues unique to the intermountain region. Participants include faculty from local universities, researchers from national laboratories, energy companies, and politicians. This year, the theme is energy security with a focus on nuclear, renewable, and alternative energy sources and continued advancements in grid and cybersecurity.
    \end{enumerate}
    \begin{enumerate}[topsep=7pt,itemsep=-0ex,partopsep=0ex,parsep=0ex,leftmargin=*,label=(\arabic*)]
        \item[]\textbf{\acl{doe} \acl{cbs} Public Meeting}
        \item[]\textit{Boise}\hfill{\textit{2016}}
        \item[]The \acl{doe} is in the initial phase of developing a consent-based process for siting the facilities needed to store and dispose of the nation’s spent nuclear fuel and high-level radioactive waste. A consent-based approach to siting relies on understanding the views of the public, stakeholders, and governments at the local, state, and tribal levels. In this first phase, \acs{doe} is engaging with interested groups and individuals to learn about what elements are important to consider in designing an enduring approach to siting.  This session is an opportunity for the public to share thoughts and perspectives through a facilitated discussion.
    \end{enumerate}
    \begin{enumerate}[topsep=7pt,itemsep=-0ex,partopsep=0ex,parsep=0ex,leftmargin=*,label=(\arabic*)]
        \item[]\textbf{Trilateral Nuclear Energy Dialogue: Korea, Japan, United States}
        \item[]\textbf{Global American Business Institute}
        \item[]\textit{Boise}\hfill{\textit{2016}}
        \item[]A high-level private workshop among preeminent Korean, Japanese, and American experts in nuclear energy and nuclear policy issues, with the intention of fostering relationships, confidence building, and seeking potential areas for trilateral cooperation. In keeping with the overarching theme of previous discussions—the back-end fuel cycle—this meeting seeks to underscore the role of advanced nuclear energy and fuel cycle technologies. Although the obstacles impeding permanent solutions to spent fuel and radioactive waste are largely political, this dialogue aims to highlight the potential of cutting-edge technologies in addressing the policy, environmental, and public acceptance challenges facing management of the nuclear fuel cycle, in addition to opportunities for international collaboration in the research, development, demonstration, and deployment of such technologies.
    \end{enumerate}
    \begin{enumerate}[topsep=7pt,itemsep=-0ex,partopsep=0ex,parsep=0ex,leftmargin=*,label=(\arabic*)]
        \item[]\textbf{Idaho's Role in Nuclear: Clean Energy Powered by the Next Generation}
        \item[]\textit{\acl{bsu}}\hfill{\textit{2016}}
        \item[]\acl{bsu} invited globally recognized leaders in nuclear energy to address the benefits and challenges associated with nuclear energy production and its role in supplying clean energy for a growing world. The summit will also include a panel discussion with John Kotek, Assistant Director of Nuclear Energy, U.S. \acl{doe}; Dr. Mark Peters, Laboratory Director for the Idaho National Laboratory; Mark Rudin, Vice President for Research at Boise State University; Mike McGough, Chief Commercial Officer at NuScale Power, and Dr. Harold Blackman, Associate Vice President for Research and Economic Development at \acl{bsu}. The panel will address questions about the benefits of and concerns around nuclear power.
    \end{enumerate}
    \begin{enumerate}[topsep=7pt,itemsep=-0ex,partopsep=0ex,parsep=0ex,leftmargin=*,label=(\arabic*)]
        \item[]\textbf{\acl{pr} \& \acl {pp} Working Group Workshop}
        \item[]\textit{University of California -- Berkeley}\hfill{\textit{2015}}
        \item[]The \acs{pr}\&\acs{pp} methodology was developed within the Generation IV International Forum to provide a structured framework to assess the proliferation resistance and physical protection robustness of advanced nuclear energy systems, and to guide designers to further improve their systems. This workshop is intended to familiarize non-experts in this field with the broad aspects of the methodology and its applications. The \acs{pr}\&\acs{pp} Working Group will present an overview of the methodology to an audience of  students, academics, and members of the community who wish to become more familiar with the methodology. To illustrate the methodology, its application to a hypothetical nuclear energy system will be examined. Workshop participants will be divided into subgroups that will consider different proliferation and security threats, and will identify important design features and approaches that contribute to the system’s resilience to these threats. Following these subgroup sessions, workshop participants will reconvene to review insights from the subgroups. Finally, an open discussion will be held to obtain feedback from the participants on the approach to \acs{pr}\&\acs{pp} with the objective of refining the methodology and its presentation to the wider community.
    \end{enumerate}
    \begin{enumerate}[topsep=7pt,itemsep=-0ex,partopsep=0ex,parsep=0ex,leftmargin=*,label=(\arabic*)]
        \item[]\textbf{Open Education Resources Development}
        \item[]\textbf{State of Idaho Board of Education}
        \item[]\textit{Boise}\hfill{\textit{2015}}
        \item[]This workshop focused on the use of open source educational materials in order to produce an online textbook as a supplement to existing commercial textbooks. An online textbook allows flexibility for faculty to augment content without requiring multiple textbooks for a course. An online textbook allows the educational content of a course to be more closely aligned with the desired learning outcomes. Two online texts have been developed -- \href{https://uidaho.pressbooks.pub/nuclearengineering}{Principles of Nuclear Engineering} and \href{https://uidaho.pressbooks.pub/riskassessment}{Risk Assessment}.
    \end{enumerate}
    \begin{enumerate}[topsep=7pt,itemsep=-0ex,partopsep=0ex,parsep=0ex,leftmargin=*,label=(\arabic*)]
        \item[]\textbf{Advanced Summer School of Radioactive Waste Disposal with Social-Scientific Literacy}
        \item[]\textit{University of California -- Berkeley}\hfill{\textit{2009; 2011}}
        \item[]\textit{Hawai'i Tokai International College }\hfill{\textit{2010}}
        \item[]This advanced summer school was established in conjunction with the Department of Nuclear Engineering \& Management at the University of Tokyo and the Department of Nuclear Engineering at the University of California--Berkeley to provide \acs{phd} students and early-career nuclear engineers with education in social sciences and engineering. The goal was to foster a next generation of engineers capable of understanding the public and societal needs, contributing to the societal decision making, and taking a responsible role as engineering experts in society. The discussant leads student group activities by stimulating questions from students and corroborating with the chair to develop a summary of lectures.
    \end{enumerate}
    \begin{enumerate}[topsep=7pt,itemsep=-0ex,partopsep=0ex,parsep=0ex,leftmargin=*,label=(\arabic*)]
        \item[]\textbf{Minner Fellows Program}
        \item[]\textit{University of California -- Berkeley}\hfill{\textit{2011}}
        \item[]The objective of this \acs{nsf} funded program was to develop a framework for making ethical judgments in engineering. Because engineering faculty and graduate students play a leadership role in the development of these technologies, it is essential that they become aware the ethical, legal and social ramifications of them. The course focused on context, or the embodiment of moral maturity and ethical expertise, in the same way that faculty and graduate students embody engineering and technical expertise.
    \end{enumerate}
    \begin{enumerate}[topsep=7pt,itemsep=-0ex,partopsep=0ex,parsep=0ex,leftmargin=*,label=(\arabic*)]
        \item[]\textbf{Summer Institute for Preparing Future Faculty}
        \item[]\textit{University of California -- Berkeley}\hfill{\textit{2005}}
        \item[]This unique program is for Doctoral Candidates with an interest in an academic career. Many aspects of teaching are covered: course design, syllabus development, teaching and learning assessment, teaching and learning strategies, and the creation of a teaching portfolio. The program exposes candidates to faculty in several disciplines both within and outside the university; thus allowing for the dissemination of the full scope of teaching methods and skills, as well as broadening of perspectives with regards to the entire teaching profession.
    \end{enumerate}
    \begin{enumerate}[topsep=15pt,itemsep=-0ex,partopsep=0ex,parsep=0ex,leftmargin=*,label=(\arabic*)]
        \item[]\textbf{Webinars \& Virtual}
        \item[]\textbf{\acl{ans}}
        \item[]\textit{Equitable Outreach: Now Comes the Hard Part}\hfill{\textit{2025}}
        \item[]\textit{Next Generation Workforce Development for Nuclear Waste Disposal}\hfill{\textit{2024}}
        \item[]\textit{Spending Time on Spent Nuclear Fuel with the \acl{doe}}\hfill{\textit{2024}}
    \end{enumerate}
    \begin{enumerate}[topsep=7pt,itemsep=-0ex,partopsep=0ex,parsep=0ex,leftmargin=*,label=(\arabic*)]
        \item[]\textbf{Idaho Section of the \acl{ans}}
        \item[]\textit{Allies for Career Success}\hfill{\textit{2024}}
    \end{enumerate}
    \begin{enumerate}[topsep=7pt,itemsep=-0ex,partopsep=0ex,parsep=0ex,leftmargin=*,label=(\arabic*)]
        \item[]\textbf{\acl{bsu} \acl{epi} Power Talks}
        \item[]\textit{Views from Former Nuclear Regulatory Commission Chairs on Siting for Used Nuclear Fuel Storage} \hfill{2024}
        \item[]\textit{Decision-making and Engagement for Used Nuclear Fuel \& Nuclear Waste} \hfill{2024}
        \item[]\textit{Energy Storage} \hfill{2023}
        \item[]\textit{Wildfire-Grid 2.0} \hfill{2023}
        \item[]\textit{Tribal Energy Decisionmaking} \hfill{2023}
        \item[]\textit{Consent-based Siting in Nuclear Energy} \hfill{2022}
        \item[]\textit{Economic Opportunities and Challenges for Idaho with Low Carbon Energy} \hfill{2022}
    \end{enumerate}
    \begin{enumerate}[topsep=7pt,itemsep=-0ex,partopsep=0ex,parsep=0ex,leftmargin=*,label=(\arabic*)]
        \item[]\textbf{\acl{wsc}}
        \item[]\textit{Western Services Corporation Generation III Generic Pressurized Water Reactor Simulator Workshop} \hfill{\textit{2022}}
    \end{enumerate}
    \begin{enumerate}[topsep=7pt,itemsep=-0ex,partopsep=0ex,parsep=0ex,leftmargin=*,label=(\arabic*)]
        \item[]\textbf{\acl{nei}}
        \item[]\textit{\acs{nei} Advanced Reactors Safeguards \& Security Workshop}\hfill{\textit{2021}}
    \end{enumerate}
    \begin{enumerate}[topsep=7pt,itemsep=-0ex,partopsep=0ex,parsep=0ex,leftmargin=*,label=(\arabic*)]
        \item[]\textbf{\acl{ui}}
        \item[]\textit{Best Practices for Working with the \acs{dod}} \hfill{\textit{2021; 2023}}
        \item[]\textit{Defense \acl{eps}\hfill{\textit{2021}}}
        \item[]\textit{\acs{nsf} Broader Impacts 101 Workshop}\hfill{\textit{2021}}
    \end{enumerate}


\end{document}
