%%%%% 
%Sam Root - University of Idaho
%Thesis defense slides with sidebar
%Customize with your school colors!
%Fork sjroot97/UI-Thesis-Dissertation for thesis guide, poster, and examples
%%%%%
\documentclass[aspectratio=169,pdftex,dvipsnames]{beamer}
\usetheme[right,hideallsubsections]{Berkeley}
\usecolortheme{seahorse}
\beamertemplatenavigationsymbolsempty 
\addtobeamertemplate{footnote}{\hskip -2em}{}


\usepackage{caption}
\captionsetup[figure]{labelformat=empty,font=small}

%packages and definitions
\usepackage{wrapfig}
\usepackage{textgreek}
\usepackage{enumitem}
\setlist[itemize]{label=\textbullet}
\usepackage{amsmath,cancel,nicefrac}
\usepackage{ulem}
\usepackage{graphicx,animate}
\graphicspath{{./img/},{}} %path to graphics
\usepackage[yyyymmdd]{datetime} %date format
\renewcommand{\dateseparator}{.}

%%%%%
\usepackage{pgf,pgfplots}
\usepackage{tikz,grffile} % required for drawing custom shapes
\usetikzlibrary{shapes,arrows,automata,trees}
\pgfplotsset{compat=newest}
%\usepgfplotslibrary{patchplots}
%%%%%

\usepackage{booktabs} % nice rules (thick lines) for tables
\usepackage{microtype} % improves typography for PDF
\usepackage{verbatim}

\usepackage[acronym,nomain,nonumberlist]{glossaries}
%\makenoidxglossaries

\usepackage{xcolor,colortbl} %change font color
\usepackage[numbers,sort&compress]{natbib} %use 'numbers' for numbered citations; 'round' for () instead [] for inline citations; nsf.bst
\usepackage{bibentry}
\usepackage[eulergreek]{sansmath}


\setlength{\bibsep}{0pt} %sets space between references
%\renewcommand{\bibsection}{} %suppresses large 'references' heading
\renewcommand\bibpreamble{\vspace{-0.2\baselineskip}} %sets spacing after heading if not using default references heading

%%%%% user commands
\newcommand\blfootnote[1]{%
  \begingroup
  \renewcommand\thefootnote{}\footnote{#1}%
  \addtocounter{footnote}{-1}%
  \endgroup
}

%Fix centering on title slide which doesn't have a sidebar [plain]
\newcommand\noside{\addtolength\textwidth{2cm} 
\setlength\hsize{\textwidth} 
\setlength\columnwidth{\textwidth}
\vfill\centering}

%Make sidebar background color so it appears invisible
\newcommand\outlineframe{
    \setbeamercolor{sidebar}{fg=Chrome,bg=Chrome}
\setbeamercolor{title in sidebar}{fg=Chrome}
\setbeamercolor{author in sidebar}{fg=Chrome}
\setbeamercolor{section in sidebar}{fg=Chrome}
}

%Make sidebar original color so it appears again
\newcommand\stopoutlineframe{
    \setbeamercolor*{sidebar}{fg=PrideGold,bg=Black}
\setbeamercolor*{title in sidebar}{fg=PrideGold}
\setbeamercolor*{author in sidebar}{fg=PrideGold}
\setbeamercolor*{section in sidebar}{fg=PrideGold}
}

\makeatletter
\setlength{\beamer@headheight}{1cm}
\renewcommand{\@biblabel}[1]{#1.\hfill} %bibliography ordered list has numbers left flush
\makeatother
\setbeamertemplate{navigation symbols}{}
% \setbeamertemplate{footline}[frame number]
\setbeamertemplate{sidebar right}[sidebar theme]
\addtobeamertemplate{sidebar right}{}{%
    \vspace{10pt}
    \hspace{20pt}
    \insertframenumber
    \vspace{10pt}
    }

% Current section in TOC
\AtBeginSection[ ]
{\outlineframe
\begin{frame}{}
    \tableofcontents[currentsection,hideothersubsections]
\end{frame}
\stopoutlineframe}

 \AtBeginSubsection[]{
     \begin{frame}{}
         \vfill
         \centering
         \begin{beamercolorbox}[sep=8pt,center,shadow=true,rounded=true]{title}
             \usebeamerfont{title}\insertsubsectionhead\par
         \end{beamercolorbox}
       
         \vfill
       
     \end{frame}
 }

\logo{\includegraphics[width=40pt]{logocorner}}

% colors
\definecolor{PrideGold}{RGB}{241,179,0}
\definecolor{Silver}{RGB}{165,169,180}
\definecolor{White}{RGB}{255,255,255}
\definecolor{Black}{RGB}{25,25,25}
\definecolor{Chrome}{RGB}{245,245,245}

\setbeamercolor{background canvas}{bg=Chrome}
\setbeamercolor{block title}{bg=Silver,fg=Black}
\setbeamercolor{block body}{bg=Silver!20,fg=Black}

\setbeamercolor{block title alerted}{bg=Black, fg=Silver}
\setbeamercolor{block body alerted}{bg=PrideGold, fg=Black}
\setbeamercolor{alerted text}{fg=Silver}

\setbeamercolor*{block title example}{bg=PrideGold, fg = Black}
\setbeamercolor*{block body example}{bg=PrideGold!20, fg = Black}

\setbeamercolor*{palette primary}{bg = PrideGold}
\setbeamercolor*{palette secondary}{bg = PrideGold, fg = White}
\setbeamercolor*{palette tertiary}{bg = PrideGold, fg = White}
\setbeamercolor*{titlelike}{fg = PrideGold}
\setbeamercolor*{title}{bg = Black, fg = PrideGold}
\setbeamercolor*{item}{fg = PrideGold}
\setbeamercolor*{caption name}{fg = PrideGold}

\setbeamercolor*{sidebar}{fg=PrideGold,bg=Black}
\setbeamercolor*{title in sidebar}{fg=PrideGold}
\setbeamercolor*{author in sidebar}{fg=PrideGold}
\setbeamercolor*{section in sidebar}{fg=PrideGold}


\setbeamercolor{section in toc}{fg=Black}
\setbeamercolor{subsection in toc}{fg=Black}

\setbeamercolor{page number in head/foot}{fg=Silver,bg=Black}
\setbeamercolor{footline}{bg=Black}

\setbeamercolor{bibliography entry author}{fg=Black}
\setbeamercolor{bibliography entry note}{fg=Black}
\setbeamercolor{bibliography entry title}{fg=Black}

% Font Information

\usefonttheme{professionalfonts}

\setbeamerfont{title}{size=\large}
\setbeamerfont{subtitle}{size=\small}
\setbeamerfont{author}{size=\small}
\setbeamerfont{date}{size=\small}
\setbeamerfont{institute}{size=\small}
\setbeamerfont{caption}{size=\tiny}

\newcommand{\edit}[1]{\textcolor{blue}{#1}} %shortcut for changing font color on revised text
\newcommand{\fn}[1]{\footnote{#1}} %shortcut for footnote tag
\newcommand*\sq{\mathbin{\vcenter{\hbox{\rule{.3ex}{.3ex}}}}} %makes a small square as a separator $\sq$
\newcommand{\sk}[1]{\sout{#1}} %shortcut for strike-through
\newcommand{\x}{\cellcolor{lightgray}\textbf{X}} %use to shade in table cell

%fix spacing for frames with subtitles
\newcommand\cST{\vspace{-0.5cm}} %Center title when no subtitle
\newcommand\ST{\vspace{-0.15cm}}  %Make subtitles fix in top bar

\newcommand{\acf}{\acrfull} %full acronym
\newcommand{\acl}{\acrlong} %long acronym
\newcommand{\acs}{\acrshort} %short acronym
\newcommand{\acfp}{\acrfullpl} %full acronym plural
\newcommand{\aclp}{\acrlongpl} %long acronym plural
\newcommand{\acsp}{\acrshortpl} %short acronym plural
\newcommand{\Acf}{\Acrfull} %full acronym first letter capital
\newcommand{\Acl}{\Acrlong} %long acronym first letter capital

\newacronym{pid}{PID}{Proportional-Integral-Derivative}
\newacronym{npp}{NPP}{Nuclear Power Plant}
\newacronym{msnb}{MSNB}{Molten Salt Nuclear Battery}
\newacronym{msr}{MSR}{Molten Salt Reactor}
\newacronym{msre}{MSRE}{Molten Salt Reactor Experiment}
\newacronym{lwr}{LWR}{Light Water Reactor}
\newacronym{smr}{SMR}{Small Modular Reactor}
\newacronym{nrc}{NRC}{Nuclear Regulatory Commission}
\newacronym{ans}{ANS}{American Nuclear Society}
\newacronym{inl}{INL}{Idaho National Laboratory}
\newacronym{nrel}{NREL}{National Renewable Energy Laboratory}
\newacronym{oak}{ORNL}{Oak Ridge National Laboratory}
\newacronym{lti}{LTI}{Linear Time-Invariant}
\newacronym{mm}{MIMO}{Multi-Input Multi-Output}
\newacronym{ss}{SISO}{Single-Input Single-Output}

\newcommand{\UF}[1][4]{$UF_{#1}$}
\newcommand{\flinak}{$FLiNaK$ }

\newcommand{\B}[1][]{$^{#1}B$ }
\newcommand{\Be}[1][]{$^{#1}Be$ }
\newcommand{\I}[1][135]{$^{#1}I$ }
\newcommand{\Xe}[1][135]{$^{#1}Xe$ }
\newcommand{\Nd}[1][149]{$^{#1}Nd$}
\newcommand{\Pm}[1][149]{$^{#1}Pm$ }
\newcommand{\Sa}[1][149]{$^{#1}Sa$ }
\newcommand{\Gd}[1][157]{$^{#1}Gd$ }
\newcommand{\U}[1][]{$^{#1}U$ }
\newcommand{\Pu}[1][239]{$^{#1}Pu$ }
\newcommand{\Ca}[1][]{$^{#1}Ca$ }
\newcommand{\Am}[1][]{$^{#1}Am$ }
\newcommand{\Po}[1][]{$^{#1}Po$ }
\newcommand{\Ra}[1][]{$^{#1}Ra$ }
%%%%%
\newcommand{\sci}[1][]{$\times$ 10\textsuperscript{#1} }
%%%%%

%Title Slide Info
\title[LaTeX Defense]{Writing a Thesis or Dissertation\\for University of Idaho with LaTeX}
\author{Joe Vandal}
\institute[Idaho Falls Center]{University of Idaho $\sq$ Idaho Falls Center for Higher Education\\
    Department of Nuclear Engineering and Industrial Management\\
    }
\date{December 6\textsuperscript{th}, 2023}    
\titlegraphic{\includegraphics[width=0.20\textwidth]{logo}}


\begin{document}
    \nobibliography*
{    \setbeamertemplate{footline}{}

%Title Slide
    \begin{frame}[plain]{}
        \noside
        \titlepage
    \end{frame}
} 

%Biography Slide
\begin{frame}{About the Author}
    \vfill
    \begin{block}{Experience}
      B.S Chemical Engineering (2015-2019) - Michigan Technological University\\
      M.S. Nuclear Engineering (2021-2023) - University of Idaho - \acs{nrc} Fellow\\
      Modeling and Simulation Intern at Idaho National Lab
    \end{block}
    \vfill
    \begin{block}{Select Publications}
    \bibentry{MyOwnPaper}\;\\
    \bibentry{RootMS}   
    \end{block}
    \vfill
\end{frame}

%Outline Slide
{\outlineframe
\begin{frame}{Outline}
    \tableofcontents[hideallsubsections]
\end{frame}
\stopoutlineframe}

\section{Introduction}
\begin{frame}{Background}
    \begin{block}{Block 1}
        \only<1>{
            \begin{itemize}
                \item Item 1
                \item Item 2
                \item Item 3
                \item Item 4
            \end{itemize}
        }
    \end{block}
    
    \begin{block}{Block 2}
        \only<2>{
            \begin{itemize}
                \item Item 1
                \item Item 2
                \item Item 3
            \end{itemize}
        }
    \end{block}
    
    \begin{block}{Block 3}
        \only<3>{
            \begin{itemize}
                \item Item 1
                \item Item 2
                \item Item 3
            \end{itemize}
        }
    \end{block}
\end{frame}

\begin{frame}{Molten Salt Nuclear Battery}
    \begin{columns}
        %Column 1
        \begin{column}{0.5\textwidth}
            \begin{block}{Left Block 1}
                \only<1>{
                    \begin{itemize}
                        \item Item 1
                    \item Item 2
                    \item Item 3
                    \end{itemize}
                }
            \end{block}
            \begin{block}{Left Block 2}
                \only<2>{
                    \begin{itemize}
                        \item Item 1
                        \item Item 2
                        \item Item 3
                    \end{itemize}
                }
            \end{block}
        \end{column}
        %Column 2
        \begin{column}{0.5\textwidth}
            \only<1>{
            \begin{figure}[!ht]
                \centering
                \includegraphics[width=0.9\textwidth]{ANS}
                \caption{ANS Logo}
                \label{fig:ans}
            \end{figure}
            }
            \only<2>{
                \begin{figure}
                    \includegraphics[width=0.9\textwidth]{logo}
                    \caption{UIdaho Logo}
                \end{figure}
            }
        \end{column}
    \end{columns}
    \end{frame}

\begin{frame}{Citations}
    \begin{block}{A Citation \cite{MyOwnPaper}}
        \only<1>{
        \begin{itemize}
            \item Item 1
            \item Item 2
        \end{itemize}    
        }
    \end{block}
    
    \begin{block}{Another Citation \cite{RootMS}}
        \only<2>{
            \begin{itemize}
                \item Item 1
                \item Item 2
            \end{itemize}    
            }
    \end{block}
    
    \begin{block}{No Citation}
        \only<3>{
            \begin{itemize}
                \item Item 1
            \end{itemize}    
            }
    \end{block}

    \blfootnote{\tiny\cite{MyOwnPaper} \tiny\bibentry{MyOwnPaper}}
    \blfootnote{\tiny\cite{RootMS} \tiny\bibentry{RootMS}}
\end{frame}

\section{Subsections}
\subsection{Subsection 1}
\begin{frame}{A Tikz Drawing}
    \begin{figure}[!ht]
        \centering
        \resizebox{\textwidth}{!}{\input{tikz/feedback}}
    \end{figure}
\end{frame}

\begin{comment}
\begin{frame}{A commented out frame}
    You can't see this
    \begin{figure}[!ht]
        \centering
        \resizebox{\textwidth}{!}{\input{tikz/feedback}}
    \end{figure}
\end{frame}
\end{comment}

\begin{frame}{An equation explained}
    \begin{equation*}
        u(t) 
        = \underbrace{K_P e(t)}_{\text{Proportional}} 
        + \underbrace{K_I \int_0^t e(t)dt}_{\text{Integral}} 
        + \underbrace{K_D \frac{de(t)}{dt}}_{\text{Derivative}}
    \end{equation*}
    \only<2->{
    \begin{block}{Proportional}
        \only<2>{
        \begin{itemize}
            \item Control output is manipulated in proportion to the error defined by the proportional gain constant
            \item High gain yields an aggressive controller that is prone to overshooting the set-point
            \item Low gain may result in steady-state offset
        \end{itemize}    
        }
    \end{block}

    \begin{block}{Integral}
        \only<3>{\begin{itemize} 
            \item Considers cumulative error to help eliminate steady-state offset 
            \item As the process variable settles around the set-point, the cumulative error approaches a constant value and the effect of the integral controller diminishes.
        \end{itemize}
        }
    \end{block}

    \begin{block}{Derivative}
        \only<4>{\begin{itemize}
            \item Estimates the time rate of change of the error to dampen overshoot
            \item Backs-off the proportional response when the process variable rapidly approaches the set-point
            \item Can be difficult to tune
        \end{itemize}
        }
    \end{block}
    }
\end{frame}


\subsection{Subheadings}
\begin{frame}{Heading\only<2->{\ST}}{\only<1>{\cST}\only<2-3>{Subheading for 2nd and 3rd slide}\only<4-5>{Subheading for 4th and 5th slide}}
    \only<1>{
         No subheading on this slide
     }
     \only<2-3>{
         \begin{block}{Block for second slide}
            \only<2>{\begin{itemize}
                \item Textbook Citation with chapter \cite[Ch. 7]{Kerlin}
                \item Another textbook citation with chapter \cite[Ch. 6]{DH}
            \end{itemize}}
         \end{block}
            
         \begin{block}{Block for third slide}
            \only<3>{\begin{itemize}
                \item A citation that only appears for the third frame \cite{MyOwnPaper}
                \item Item 2
            \end{itemize}}
         \end{block}

         \only<2>{\blfootnote{\tiny\cite{Kerlin} \tiny\bibentry{Kerlin}}
         \blfootnote{\tiny\cite{DH} \tiny\bibentry{DH}}}
         \only<3>{\blfootnote{\tiny\cite{PetersonMS} \tiny\bibentry{PetersonMS}}}
     }
     \only<4-5>{
             \begin{block}{Block for 4th slide}
                \only<4>{\begin{itemize}
                    \item Emphasis \textit{emphasis}
                    \item Item 2
                    \item $t_{\nicefrac{1}{2}}$ halflife - bring back the citation from the second slide \cite[Ch. 6]{DH}
                \end{itemize}
                \begin{equation*}
                    {^{87}Br} \underset{56 sec}{\stackrel{\beta^-}{\longrightarrow}} {^{87}Kr^{*}} \to {^{86}Kr + n}
                \end{equation*}}
             \end{block}
             
             \begin{block}{Block for fifth slide}
                \only<5>{\begin{itemize}
                    \item Precursors produced near the core exit and long lived precursors may emit their neutrons outside of the core 
                    \item These neutrons are effectively lost from the fission chain reaction \cite[Ch. 3]{Kerlin}
                    \item Larger power transport requires a higher flow rate
                    \item Greater delayed neutron losses 
                    \item Negative feedback
                \end{itemize}}
             \end{block}
              

             \only<4>{\blfootnote{\tiny\cite{DH} \tiny\bibentry{DH}}}
             \only<5>{\blfootnote{\tiny\cite{Kerlin} \tiny\bibentry{Kerlin}}}
     }
 \end{frame}

\begin{frame}{A Table}
    \begin{table}[bh!]
        \centering\footnotesize
        \begin{tabular}{c|ccccc|cc}
        \hline
        Years & c\textsubscript{4} \sci[9] & c\textsubscript{3} \sci[6] & c\textsubscript{2} \sci[4] & c\textsubscript{1} \sci[2] & c\textsubscript{0} & root ($^o$) & slope ($pcm/^o$) \\ \hline
        0.0 & -2.797 & 1.789 & -4.361 & 4.829 & -2.009 & 111.41 & 224.24 \\
        0.5 & -2.755 & 1.755 & -4.272 & 4.732 & -1.976 & 113.69 & 203.69 \\
        1.0 & -1.838 & 1.253 & -3.253 & 3.826 & -1.682 & 115.79 & 189.71 \\
        1.5 & -2.533 & 1.632 & -3.253 & 4.507 & -1.909 & 117.38 & 175.96 \\
        2.0 & -2.418 & 1.578 & -3.930 & 4.440 & -1.895 & 119.45 & 161.06 \\\hline
        2.5 & -1.461 & 1.026 & -2.750 & 3.337 & -1.515 & 121.06 & 152.71 \\
        3.0 & -1.137 & 0.856 & -2.425 & 3.070 & -1.440 & 122.67 & 146.08 \\
        3.5 & -2.054 & 1.357 & -3.433 & 3.953 & -1.727 & 124.58 & 130.85 \\
        4.0 & -2.527 & 1.617 & -3.967 & 4.438 & -1.892 & 126.46 & 120.67 \\\hline
        4.5 & -2.869 & 1.831 & -4.460 & 4.935 & -2.081 & 128.35 & 111.30 \\
        5.0 & -2.338 & 1.520 & -3.785 & 4.291 & -1.855 & 130.55 & 102.04 \\
        5.5 & -1.471 & 1.054 & -2.852 & 3.467 & -1.585 & 132.29 & 93.34 \\
        6.0 & -2.626 & 1.702 & -4.211 & 4.729 & -2.027 & 134.96 & 83.90 \\\hline
        6.5 & -1.672 & 1.141 & -2.985 & 3.550 & -1.607 & 137.45 & 77.56 \\
        7.0 & -3.321 & 2.095 & -5.036 & 5.492 & -2.292 & 139.19 & 69.73 \\
        7.5 & -2.419 & 1.579 & -3.936 & 4.459 & -1.932 & 142.69 & 59.45 \\
        8.0 & -7.991 & 0.648 & -1.960 & 2.623 & -1.305 & 144.62 & 55.65 \\
        \end{tabular}
        \label{tab:Depletionfit}
    \end{table}
\end{frame}



\section{Conclusions}
\begin{frame}{Conclusions}
    \begin{block}{Summary of Work Completed}
        \only<1>{
            \begin{itemize}
                \item Item 1
                \item Item 2
                \item Item 3
            \end{itemize}
        }
        \begin{block}{Results In-Brief}
            \only<2>{
                \begin{itemize}
                    \item Item 1
                    \item Item 2
                    \item Item 3
                \end{itemize}
            }
        \end{block}
    \end{block}
\end{frame}

\begin{frame}{Acknowledgements}
    \centering
    This work and my coursework is being completed under a Graduate Fellowship funded by \acf{nrc}.
\end{frame}

\begin{frame}[plain]{}
    \noside
    \includegraphics[height=0.9\textheight]{logo}
    \vfill
\end{frame}

\begin{frame}[allowframebreaks]{References}
    \bibliographystyle{./rcs/neup}
    \footnotesize
    \bibliography{./rcs/References}
\end{frame}

%%%%%%%%%%%%%%%%%%%%%%%%%%%%%%%%%%%%%%%%%%%%%%%%%%%%%%%%%%%%%%%%%%%%%%%%%%%%%%%%%%%%%%%%%%%%%%%%%%%%%%%%%%%%%%%%%%%%%%%%%%%%%%%%%%%%%%%%%%%%%%%%%%%%%%%%%%%%%%%%%%%%%%%%%%%%%%%%%%%%%%%%%%%%%

\end{document}


